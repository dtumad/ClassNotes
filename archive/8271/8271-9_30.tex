%&pdflatex
\documentclass[11pt]{article}

\usepackage{main-macros}

\usepackage{newpxmath}
\usepackage{newpxtext}
\usepackage[margin=0.75in]{geometry}


\title{8271 ?/? Reading}
\author{Devon Tuma}
\date{Fall 2020}

\begin{document}
\maketitle

\section*{Question Answering}

\begin{itemize}
\item [1] What is Intel MPK? Which MPK-related registers are used to control memory permissions?

  MPK allows partitioning memory by assosiating virtual pages with different protection keys to create distinct domains.
  The access permisions for these domains are controlled by the PKRU register, controlled by the fast WRPKRU instruction.
  The EAX register can also control how the PKRU register changes when XRSTOR instruction is called.
  
\item [2] How does ERIM prevent untrusted code from abusing the instructions that can manipulate memory permissions?

  ERIM uses inspection of the binary code to examine each instance of a WRPKRU or XRSTOR instruction to determine if they are safe.
  A WRPKRU instance is deemed safe if it is followed by either reserved memory access or a sequence of instuctions that checks the permissions in the PKRU register set by WRPKRU.
  A XRSTOR instance is deemed safe if it is followed by a check to the EAX bit that controlls whether XRSTOR affects PKRU.
  
\end{itemize}

\section*{Paper Critiques}

\subsection*{Short Summary}

The paper discusses ERIM, a method for isolating memory by partitioning virtual pages based on the PKRU register provided by intel MPK.
ERIM then provides call gates that can be used by programs to access isolated areas of memory safely.
This seperation is enforced by binary inspection of the code to determine whether all instructions that affect this register are safe from exploitation by an attacker.
ERIM also provides a systemic way to modify binary code in order to make it safe from exploitation.
Alternatively, it is possible to wrap two seperate dynamically linked libraries in code to make it compatible with the ERIM system as wel.

\subsection*{Limitations of the paper}

\subsection*{Solutions to the limitations}

\subsection*{Potential follow-up work}

One potential piece of follow-up work would be to build an analyzer that can determine whether the private stacks in a multi-threaded application are properly allocated for compatibility with ERIM, since improper allocations can still compromise security in a multi-threaded ERIM system.

\end{document}
