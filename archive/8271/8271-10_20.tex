%&pdflatex
\documentclass[11pt]{article}

\usepackage{main-macros}

\usepackage{newpxmath}
\usepackage{newpxtext}
\usepackage[margin=0.75in]{geometry}


\title{8271 10/21 Reading}
\author{Devon Tuma}
\date{Fall 2020}

\begin{document}
\maketitle

\section*{Question Answering}

\begin{itemize}
\item [1] How did the authors evaluate the validity of predicted scenarios and what are the results?

  The authors used external evaluators to determine validity of the predicted scenarios, using 16 external people that had not been involved at all in the creation of the scenarios.
  The first part of this was to do a comparison study, to test whether the system can generate routines that appear equally valid to real routines.
  They found that over half of users rated all of the scenarios as strongly valid, and for the up/4-grams the generated scenarios did even better than the real ones.
  This gives very positive results for the generated n-grams being similarly valid to real user scenarios.
  Also the unnatural routines were often rated as invalid and sometimes even unsafe.
  Additionally, a sequence generation study was performed.
  The system still performed well enough in this case, even when evaluators provided history values that were not in the training data set.
  
\item [2] Describe three most important lessons your learnt from this paper regarding smart home.

  Firstly, smart homes place significant importance on adaptiveness, to a much greater extent that I had realized, which creates many privacy concerns but can also allow for much greater privacy as well.
  Secondly, the importance of trying to mitigate the functionality-privacy trade-off when designing smart home systems.
  Finally I hadn't previously thought about the privacy concerns that come from collecting logging information from real smart-home systems.

  
\end{itemize}

\section*{Paper Critiques}

\subsection*{Short Summary}

The paper discusses a language model system for generating realistic sequences of events in a smart-home system.
These can be used both in desiging and testing a variety of sytems.
The authors also use real trials to determine the validity of the designed system, and find that it is rated highly in comparison to manually created user scenarios.

\subsection*{Potential follow-up work}

One potential follow-up work would be finding a way to improve the sequence generation capabilities (as opposed to the total n-gram generation).
The validity studies found that users were less in agreement about the correctness on this front, so it could be a good area of improvement.

\end{document}
