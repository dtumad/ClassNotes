%&pdflatex
\documentclass[11pt]{article}

\usepackage{main-macros}

\usepackage{newpxmath}
\usepackage{newpxtext}
\usepackage[margin=0.75in]{geometry}


\title{8202 HW6}
\author{Devon Tuma}
\date{Spring 2020}

\begin{document}
\maketitle

\problem{10.1.18}
\begin{proof}
  We know that the $F[x]$-submodules for this $T$ are exactly the $T$-stable subspaces of $V$.
  But note that for any $v \in V$, $Tv \not \in V$, since $v$ and $Tv$ are orthogonal by construction.
  Hence any $1$-dimensional subspace of $V$ is not $T$-stable, and so isn't an $F[x]$-submodule for this $T$.
  Then the only other subspaces are the $0$-dimensional subspace $0$ and the $2$ dimensional subspace $V$.
  Both are trivially $T$-stable, so we conclude that these are the only $F[x]$-submodules for this $T$.
\end{proof}


\problem{10.1.19}
\begin{proof}
  We know that the $F[x]$-submodules for this $T$ are exactly the $T$-stable subspaces of $V$.
  Furthermore $V$, $0$, the $x$-axis, and the $y$-axis are all trivially $T$ stable, so we just need to check no other subspaces are $T$-stable.
  All other subspaces are $1$ dimensional, so consider the subspace with basis $\{v\}$, where $v$ doesn't lie on either axis.
  Then $Tv$ does lie on the $y$-axis, so $Tv$ generates a different subspace than $v$, and hence this isn't $T$-stable.
  Finally then we conclude that the $4$ subspaces earlier are exactly the $F[x]$ submodules for this $T$.
\end{proof}


\problem{10.1.20}
\begin{proof}
  We know that the $F[x]$-submodules for this $T$ are exactly the $T$-stable subspaces of $V$.
  Hence it suffices to show that any subspace of $V$ is $T$-stable.
  But note that $T$ sends vectors to their negative (i.e. their inverse), and subspaces are by definition closed under inverses, so every subspace is $T$-stable and hence an $F[x]$-submodule for this $T$ as desired.
\end{proof}


\problem{10.2.5}
\begin{proof}
  Note that $\Z/30\Z$ is generated by $1$, and so any $\varphi \in \text{Hom}(\Z/30\Z, \Z/21\Z)$ is determined by the image $\varphi(1)$.
  By Lagrange, we know that $\varphi(1)$ must have order dividing $21$.
  Furthermore, the order of $\varphi(1)$ must divide the order of $1$ in $\Z/30\Z$, so it must divide $30$.
  Hence the order of $\varphi(1)$ is either $1$ or $3$.

  If it has order $1$, then we get $\varphi(1) = 0$, and if it has order $3$ then $\varphi(1) \in \{7,14\}$ since if $21$ divides $3k$ then $7$ must divide $k$.
  These three options give distinct homomorphisms since the image of $1$ is different in each case, so we conclude that there are exactly three such $\Z$-module homomorphisms.
\end{proof}


\problem{10.3.9}
\begin{proof}
  For the forward direction, assume that $M$ is irreducible, so in particular $M \ne 0$ and we can pick $m \in M$ such that $m \ne 0$.
  Then $(m)$ is a non-zero submodule of $M$, and so since $M$ is irreducible it must be that $M = (m)$, making $M$ cyclic.
  For the reverse direction, assume $M \ne 0$ is cyclic with any non-zero element as a generators.
  Then for any non-zero submodule $N$ of $M$, if we pick a non-zero $n \in N$ then by assumption $n$ is a generator so $M = (n) \subseteq N$, which implies that $M = N$.
  Hence the only non-zero submodule of $M$ is itself, so we conclude that $M$ is irreducible.

  Then any irreducible $\Z$ module must be cyclic with no non-trivial submodules.
  All infinite cyclic groups are isomorphic to $\Z$ itself, which has non-trivial subgroups like $3\Z$, so any irreducible $\Z$ module must be a finite cyclic group, and hence isomorphic to $\Z/n\Z$ for some $n \in \N$.
  But then we know that a cyclic group or order $n$ has subgroups of order $d$ for each $d$ dividing $n$, so we conclude that the only irreducible $\Z$ modules are those isomorphic to $\Z/p\Z$ for $p$ a prime.
\end{proof}

\problem{10.3.10}
\begin{proof}
  Assume $R$ is commutative, and let $M$ be an $R$-module.
  For the forward direction, assume that $M$ is irreducible, so in particular $M \ne 0$ and we can pick a non-zero $m \in M$.
  Then consider the map $\varphi: R \rightarrow M$ given by $r \mapsto rm$.
  The image of this map is not the zero module since $1$ maps to $m \ne 0$, so since $M$ is irreducible, the image must be all of $M$.
  Hence by the first isomorphism theorem $R/\ker(\varphi) \cong M$, and it just remains to show $\ker(\varphi)$ is maximal.
  But note that any submodule containing $\ker(\varphi)$ would correspond to a submodule of $R/\ker(\varphi)$ and hence a submodule of $M$.
  Therefore since $M$ is irreducible, it must correspond to either $0$ or $M$, and hence be trivial, and so $\ker(\varphi)$ is indeed maximal.

  For the reverse direction, assume that $M$ is isomorphic to $R/I$ for some maximal ideal $I$.
  Then the quotient by a maximal ideal gives a field, and the only subfields of a field are the trivial ones, so the only submodules of $M$ will be the trivial ones, making $M$ irreducible as desired.
\end{proof}


\problem{10.3.11}
\begin{proof}
  Let $M_1$ and $M_2$ be two irreducible $R$-modules, and let $\varphi$ be a non-zero homomorphism between them.
  $\ker(\varphi)$ is a submodule of $M_1$, so must be either $0$ or $M_1$ since by irreducibility.
  However it can't be $M_1$ since $\varphi$ is a non-zero map, so $\ker(\varphi) = 0$ and hence $\varphi$ is injective.
  Furthermore, the image of $\varphi$ is either $M_2$ or $0$ by irreducibility, and so must be $M_2$ since $\varphi$ is non-zero.
  Finally then this implies that $\varphi$ is surjective as well, and we conclude that it is an isomorphism.
\end{proof}


\problem{12.1.1}
\subproblem{a}
\begin{proof}
  Let $M$ be a module over the integral domain $R$, and let $x$ be a non-zero torsion element, so $rx = 0$ for some $r \in R$.
  Then we have $r*x + 0*0 = r*x = 0$, so that $x$ and $0$ are linearly dependent.
  Hence any set of non-zero vectors in $\text{Tor}(M)$ has a dependence and so isn't linearly independent.
  Therefore we conclude that the rank of $\text{Tor}(M)$ is $0$.
\end{proof}

\subproblem{b}
\begin{proof}
  First, any set of linearly independent elements of $M/\text{Tor}(M)$ extends to a linearly independent set of elements of $M$ by just picking a representative for each of the cosets, so the rank of the quotient is at most the rank of $M$.
  But also, any two elements of $M$ that project to the same coset of $\text{Tor}(M)$ have a linear dependence, since their difference is a torsion element, so can be scaled to $0$.
  Hence any set of linearly independent elements of $M$ projects to a set of equal size in $M/\text{Tor}(M)$, which is also still linearly independent since any dependence would extend up by just subtracting off the coset.
  Therefore the rank of $M$ is at most the rank of $M/\text{Tor}(M)$, and so we conclude that the two must have the same rank.
\end{proof}


\problem{12.1.2}
\subproblem{a}
\begin{proof}
  Assume that $M$ has rank $n$ and that $x_1, \dots, x_n$ is a maximal set of linearly independent elements of $M$.
  Let $e_1, \dots, e_n$ be the standard basis for $R^n$, and consider the map $\varphi$ defined by $e_i \mapsto x_i$.
  This map has trivial kernel, since the image of any linear combination of the $e_i$ is a linear combination of the $x_i$, which are by definition linearly independent so this can only be $0$ if the original combination was $0$.
  Hence $R^n/\ker(\varphi) \cong R^n/0 = R^n$, so by the first isomorphism theorem $R^n$ is isomorphic to the image of $\varphi$.
  But the image of this map exactly $N$ by construction, since $N$ is exactly the linear combinations of the $x_i$, so $R^n \cong N$.

  It then remains to show that $M/N$ is a torsion $R$-module, so consider some $x \in M$.
  $x,x_1,\dots,x_n$ must have a dependence since the $x_i$ were assumed to be a maximal set, so we can write $ax = a_1x_1 + \dots + a_nx_n$ for some constants $a,a_1,\dots,a_n$ not all zero.
  Furthermore $a$ can't be zero since the $x_i$ are independent.
  But then $ax \in N$ by definition of $N$, so $a(x + N) = 0 + N$ in $M/N$, and hence $x+N$ is a torsion element.
  $x$ was arbitrary in $M$, so all elements of $M/N$ are torsion as desired.
\end{proof}

\subproblem{b}
\begin{proof}
  Assume that $M$ contains a submodule $N$ that is free of rank $n$ such that $M/N$ is a torsion $R$-module.
  Then $M$ must have rank at least $n$ since it contains $N$, so it suffices to show that any $n+1$ elements of $M$ have a dependence.
  So let $y_1,\dots,y_{n+1} \in M$, and since $M/N$ is torsion we have that for each $i$ there exists non-zero $r_i$ such that $r_iy_i \in N$.
  But then $N$ is of rank $n$, so there is a dependence among the set $r_1y_1,\dots,r_ny_n$.
  Hence we have that $a_1r_1y_1 + \dots + a_nr_ny_n = 0$ for some $a_i$.
  But this is also a dependence among the $y_i$, since the $r_i$ aren't zero divisors by the fact $R$ is an integral domain.
  Therefore we have a dependence among any $n+1$ elements, and conclude the rank of $M$ is only $n$ as desired.
\end{proof}


\problem{12.1.3}
\begin{proof}
  Let $R$ be an integral domain and let $A$ and $B$ be $R$-modules of ranks $m$ and $n$, respectively. Let $M$ denote $A \oplus B$.
  We can find $a_1, \dots, a_m$ linearly independent in $A$, and $b_1, \dots, b_n$ linearly independent in $B$.
  Then $(a_1,0), \dots, (a_m,0), (0,b_1), \dots, (0,b_n)$ are independent, since any dependence would give a component-wise dependence, and hence a dependence among the $a_i$ and the $b_j$.
  Hence the submodule $N$ generated by these elements is free of rank $n+m$, so by the previous exercise it suffices to check $M/N$ is torsion.

  Consider then an element $(a,b) \in M/N$.
  $a$ is a linear combination of the $a_i$, so by the previous exercise $a$ is torsion in the submodule generated by the quotients of the $a_i$, so we can write $ra = r_1a_1 + \dots + r_ma_m$ for some $r,r_1,\dots,r_m \in R$ not all zero.
  For the same reason we can write $sb = s_1b_1 + \dots + s_nb_n$ for some $s,s_1,\dots,s_n \in R$ not all zero.
  But then $rs(a,b) = (s(ra),r(sb))$ is a linear combination of the elements of $N$ in the obvious way, so $(a,b)$ is torsion in $M/N$.
  Finally then we conclude that $M = A \oplus B$ is of rank $n + m$ exactly as desired.
\end{proof}


\problem{12.1.4}
\begin{proof}
  Let $R$ be an integral domain, and $M$ be an $R$-module with $N$ a submodule of $M$.
  Suppose also that $M$ has rank $n$, $N$ has rank $r$, and $M/N$ has rank $s$.
  Then let $x_1,\dots,x_s$ be coset representatives of a maximal independent set in $M/N$.
  Furthermore let $x_{s+1},\dots,x_{s+r}$ be a maximal set of independent elements in $N$.
  Then assume for contradiction that the set $x_1,\dots,x_{s+r}$ has a linear dependence.
  Such a dependence must include some $x_i$ with $i < s+1$ since otherwise it would be a dependence among just the elements of $N$, and we assumed those were dependent.
  But then taking the projection of this dependence, $x_i$ doesn't project to $0$ since $i < s+1$, and so we get a dependence among the cosets of $x_1,\dots,x_s$, contradicting the assumption these are independent.
  Therefore we conclude that $x_1,\dots,x_{s+r}$ is a linearly independent set, and we get $n \ge s + r$.

  Furthermore, for any element $y \in M$, we can express the projection of $y$ to $M/N$ as a combination of the $x_1,\dots,x_s$.
  But then by the same method as in 12.1.2, if we add in the projection of another $x \in M$ then when we extend back to $M$ with the $x_{s+1},\dots,x_{s+r}$ we won't have all zero entries since we don't have zero divisors in $R$, so we'll have a dependence among the $s+r+1$ elements.
  Therefore it isn't possible to have more than $s+r$ independent elements of $M$, and we get $n \le s + r$.
  Altogether then we have $n = s+r$, which is what we wanted to show.
\end{proof}


\problem{12.1.6}
\begin{proof}
  Let $R$ be an integral domain and $M$ be a non-principle ideal of $R$.
  First $M$ has rank at most $1$, since for any two elements $x,y$ we have the dependence $x(y) - y(x) = 0$.
  But also note that for any $x \in M$ and $a \in R$ both non-zero, $ax \ne 0$ since $R$ is an integral domain.
  Therefore the rank is at least $1$ since no element alone has a dependence, and $M$ is torsion free since no elements have an annihilator.
  So $M$ is indeed torsion free of rank $1$.
  Finally if $M$ were free then since it's rank $1$ it would be generated by a single element, which would contradict it being non-principle.
\end{proof}


\end{document}
