%&pdflatex
\documentclass[11pt]{article}

\usepackage{main-macros}


\usepackage[margin=0.75in]{geometry}


\title{8202 Midterm 2}
\author{Devon Tuma}
\date{Spring 2020}

\begin{document}
\maketitle

\problem{1}
\subproblem{a: True}
\begin{proof}
  Let $\K$ be the splitting field over $\F$ for some finite list of separable polynomials $\{f_1,\dots,f_n\}$ in $\F[x]$.
  Take $f(x) = \lcm_{1 \le i \le n}(f_i)$, which is well defined and in $\F[x]$ since $\F[x]$ is a UFD.
  Note that $f$ is separable, since by the separability of the $f_i$ each of them contains at most $1$ power of any linear factor $(x-\alpha)$.
  Then we claim $\K$ is the splitting field for the separable polynomial $f(x)$.
  
  First, any linear factor of $f$ is also a factor of some $f_i$ (Or else we could remove it to get a smaller lcm), so any root of $f$ is a root of some $f_j$.
  But $\K$ splits $f_j$ by definition, so $\K$ contains every root of $f$, and hence $f$ splits over $\K$.
  Furthermore, since $\K$ is the splitting field for the set of $f_i$, any proper subfield of $\K$ must have some $f_j$ that doesn't split.
  In particular, there is some root $\alpha'$ of some $f_j$ that isn't in the subfield.
  But then $(x-\alpha')$ is a factor of $f_j$, so it's also a factor of $f$, and so $\alpha'$ is a root of $f$ not in the subfield.
  Hence $f$ doesn't split over any proper subfield of $\K$.
  Therefore $\K$ is the smallest field over which $f$ splits, and we conclude $\K$ is the splitting field for $f(x)$ in $\F[x]$.
\end{proof}

\subproblem{b: True}
\begin{proof}
  Let $\K$ be the splitting field over $\F$ for some possibly infinite list of separable polynomials $P = \{f_i\}_{i \in I}$ in $\F[x]$.
  Furthermore, assume that $\K/\F$ is finite.
  By the previous problem it suffices just to consider the case where $I$ is infinite.
  Therefore we can let $I_0 \subset I_1 \subset I_2 \subset \dots$ be an strictly increasing sequence of proper finite subsets of $P$, satisfying $\cup_{j \in \N} I_j = I$ (This only works for countable $I$, but I couldn't come up with a proof for the uncountable case).

  Then consider the sequence $\K_0 \subseteq \K_1 \subseteq \dots$, where $\K_j$ is the splitting field of $I_j$.
  Note that $[\K_j:\F] \le [\K:\F]$, so since $\K/\F$ is finite, the sequence $[\K_0:\F], [\K_1:\F], \dots$ is bounded.
  This sequence is non-decreasing by construction, so it's a bounded non-decreasing sequence of integers, and hence eventually constant.
  Therefore by multiplicativity of extension degrees, $[\K_{j+1}:\K_j]$ is eventually $1$.
  Hence there is some $\K_n$ after which the sequence $\{\K_j\}$ is constant %, and we therefore get $\cup_{j \in \N} \K_j = \K_n$.
  But then for any $m \ge n$, $\K_n = \K_m$ and so $\K_n$ is the splitting field for each $I_m$.
  Therefore $\K_n$ is the splitting field for the union $\cup_{j \in \N} I_j$, which we assumed was $I$.
  Hence by the uniqueness of splitting fields, we have $\K = \K_n$.
  Therefore $\K$ is the splitting field for the finite collection of separable polynomials $I_n$, and so by the previous problem we're done.
\end{proof}

\subproblem{c: False}
\begin{proof}
  Let $\K/\F$ be any Galois extension, so it is the splitting field for some separable polynomial $g(x)$.
  Then it is also the splitting field for the inseparable polynomial $f(x) = g(x)^2$, since they have the same set of roots.
\end{proof}

\subproblem{d: False}
\begin{proof}
  We know that we can factor $x^n - 1$ as $x^n - 1 = \prod_{d | n} \Phi_d(x)$.
  Furthermore we have $n = 10^9 = 2^95^9$, so $n$ has $(9+1)(9+1) = 100$ divisors.
  So we get at least $100$ factors determined by the $\Phi_d$ for $d$ dividing $n$.
  But some of the $\Phi_d$ can further factor over $\R$, since for example we can factor $\Phi_{10}$ as
  \begin{equation*}
    \Phi_{10}(x) = x^4-x^3+x^2-x+1
    = \frac{1}{4}(2x^2 - \sqrt{5}x - x + 2)(2x^2 + \sqrt{5}x - x + 2)
  \end{equation*}
  Therefore we have strictly more than $100$ irreducible factors of $x^n - 1$.
\end{proof}

\subproblem{e: True}
\begin{proof}
  We can still factor $x^n - 1$ as $x^n - 1 = \prod_{d | n} \Phi_d(x)$.
  But then the $\Phi_d$ are primitive polynomials that are irreducible over $\Z$, so they are all irreducible over $\Q$.
  So we have exactly $100$ irreducible factors of $x^n-1$ corresponding to the $100$ divisors of $n$.
\end{proof}

\subproblem{f: False}
\begin{proof}
  Take $\F = \F_p(t)$ for some prime $p > 2$, and consider the extension $\F_p(t, t^{1/p}) = \F_p(t^{1/p})$.
  This extension has degree $p$, and has $\F_p(t^{2/p}) \ne \F_p(t^{1/p})$, since we have no relations to use to write $t^{1/p}$ as a rational function of $t^{2/p}$.
  But since $p$ is prime we also have $\gcd(2,p) = 1$, so the claim is false.
\end{proof}


\problem{2}
\begin{proof}
  Let $f(x)$ be irreducible in $\F[x]$ with degree $k$, and let $[\K:\F] = n$.
  Taking the contrapositive of the claim, it suffices to assume that $f(x)$ reduces in $\K[x]$ and show $\gcd(k,n) > 1$.
  So assume $f$ reduces over $\K$.

  Next let $\alpha$ be a root of $f$ in some extension (not necessarily $\K$).
  Since $f$ reduces over $K$, the minimal polynomial for $\alpha$ over $\K$ must be a strict divisor of $f$.
  Therefore $m_{\alpha, \K}(x)$ has degree strictly less than $k$, so $[\K(\alpha) : \K] < \deg(f) = k$.
  Hence $[\K(\alpha) : F] = [\K(\alpha) : \K][\K : \F] < kn$.

  Then note that both $\K$ and $\F(\alpha)$ are intermediate subfields between $\F$ and $\K(\alpha)$. Using the first as an intermediate in a tower we get
  \begin{equation*}
    [\K(\alpha) : \F] = [\K(\alpha) : \K][\K : \F] = n[\K(\alpha) : \K]
  \end{equation*}
  and using the second as an intermediate (and noting that $m_{\alpha, \F}(x) = f(x)$ by the irreducibility of $f$) we get
  \begin{equation*}
    [\K(\alpha) : \F] = [\K(\alpha) : \F(\alpha)][\F(\alpha) : \F] = k[\K(\alpha) : \F(\alpha)]
  \end{equation*}
  So we see that both $k$ and $n$ divide $[\K(\alpha) : F]$, and get that $\lcm(k,n)$ divides $[\K(\alpha) : \F]$.
  But we already saw $[\K(\alpha) : \F] < kn$, so we have $\lcm(k,n) < kn$.
  Finally this implies $\gcd(k,n) = kn/\lcm(k,n) > 1$, which is what we needed to show
\end{proof}


\problem{3}
\subproblem{a: Yes}
\begin{proof}
  $K$ is the $100$th cyclotomic extension, which is the splitting field for the $100$th cyclotomic polynomial $\Phi_{100}(x)$.
  The roots of this are by definition the distinct primitive $100$th roots of unity so $\Phi_{100}(x)$ is separable.
  Hence $K$ is the splitting field for a separable polynomial and is therefore Galois.
\end{proof}

\subproblem{b: 40}
\begin{proof}
  We know that $\Phi_{100}(x)$ is the minimal polynomial for $\zeta_{100}$, and $\deg(\Phi_{100}(x)) = \varphi(100) = 40$, so $[\K:\Q] = 40$.
\end{proof}

\subproblem{c: $Aut(\K/\Q) \cong \Z/2\Z \times \Z/20\Z$}
\begin{proof}
  First, elements $\sigma \in Aut(\K/\Q)$ are determined by their action on the generator $\zeta_{100}$.
  Furthermore, $\sigma(\zeta_{100})$ must be another root of its minimal polynomial $\Phi_{100}(x)$.
  But then $\sigma(\zeta_{100})$ must be another primitive root of unity, since those are all the roots of $\Phi_{100}(x)$.
  Hence $\sigma(\zeta_{100}) = \zeta_{100}^k$ for some $k < 100$ that is relatively prime to $100$.

  Composing these automorphisms then amounts to multiplying the corresponding exponents, since $(\zeta_{100}^{k_1})^{k_2} = \zeta_{100}^{k_1k_2}$.
  Furthermore, we can do this multiplication of exponents modulo $100$ since $\zeta_{100}$ has order $100$.
  Therefore $Aut(\K/\Q)$ is isomorphic to the group of integers relatively prime to $100$ under multiplication modulo $100$,
  so we conclude that $Aut(\K/\Q) \cong (\Z/100\Z)^\times$

  Then by Sun Zi's theorem, $(\Z/100\Z)^\times \cong (\Z/2^2\Z)^\times \times (\Z/5^2\Z)^\times$.
  We then have $(\Z/4\Z)^\times \cong \Z/2\Z$, since that's the only group of order $\varphi(4) = 2$.
  Also $(\Z/25\Z)^\times \cong \Z/20\Z$, by noting that $\varphi(25) = 20$ and manually checking that $2$ has order $20$.
  Finally then we conclude that $Aut(\K/\Q) \cong (\Z/100\Z)^\times \cong \Z/2\Z \times \Z/20\Z$.
\end{proof}

\subproblem{d: 16}
\begin{proof}
  By the Fundamental Theorem of Galois Theory it suffices to count the subgroups of $G = \Z/2\Z \times \Z/20\Z$.
  The subgroups of $\Z/20\Z$ are all of the form $\gen{d}$ for some $d$ dividing $20$.
  $20$ has exactly $6$ divisors, so $\Z/20\Z$ has $6$ subgroups.

  Then we claim the subgroups of $G$ are of the form $\gen{(0,d)}$, $\gen{(1,d)}$, or $\gen{(0,d),(1,d)}$ for some $d$ dividing $20$.
  This follows from the fact that the projection of any subgroup back to $\Z/20\Z$ must be a subgroup.
  Since $20$ has exactly $6$ divisors, this naively gives $18$ subgroups.
  But note that if $20/d$ is odd, then $20/d*(1,d) + (1,d) = (0,d)$, so $\gen{(1,d)}$ and $\gen{(0,d),(1,d)}$ are the same group in this case.
  This happens for exactly $2$ divisors of $20$, which means we over-counted $2$ subgroups, and conclude there are actually $16$ subgroups.
\end{proof}

\subproblem{e: All}
\begin{proof}
  Since the Galois group is Abelian, every subgroup of $Aut(\K/\Q)$ is normal, and so by the Fundamental Theorem of Galois Theory every subfield of $\K$ is Galois over $\Q$.
\end{proof}

\subproblem{f: 2}
\begin{proof}
  By the Fundamental Theorem of Galois Theory, it suffices to count which subgroups of $Aut(\K/\Q)$ fix $\alpha := \zeta + \zeta^{-1}$.
  By the previous characterization, any automorphism in $Aut(\K/\Q)$ sends $\alpha$ to $\zeta^k + \zeta^{-k}$ for some $1 \le k \le 100$.
  This fixes $\alpha$ only when $k = 1$ or $k = \equivmod{99}{-1}{100}$.
  These two elements generate only two different subgroups, $\{1\}$ and $\{1,\sigma_{99}\}$.
  Therefore we conclude that the only two subfields containing $\alpha$ are the two corresponding fixed fields for these subgroups.
\end{proof}


\problem{4}
\begin{proof}
  Assume that $f(x),g(x)$ lie in $\F[x]$ with $f(x)$ irreducible of degree $n$.
  Then let $\alpha$ be a root of the composite $f \circ g$, so $g(\alpha)$ is a root of $f$.
  But since $f$ is irreducible it must then be the minimal polynomial for $g(\alpha)$, which implies $[\F(g(\alpha)) : \F] = n$.

  But then note that $g(\alpha) \in \F(\alpha)$, since the coefficients of $g$ lie in $\F \subset \F(\alpha)$.
  Therefore $\F(g(\alpha)) \subseteq \F(\alpha)$, and we get
  \begin{equation*}
    [\F(\alpha) : \F] = [\F(\alpha) : \F(g(\alpha))][\F(g(\alpha)) : \F] = n[\F(\alpha) : \F(g(\alpha))]
  \end{equation*}
  Hence $n$ divides $[\F(\alpha) : \F] = \deg(m_{\alpha, \F}(x))$.
  But then any irreducible factor of $f(g(x))$ is the minimal polynomial for any of its roots, so applying the previous argument to any such root shows that $n$ divides the degree of any irreducible factor.
\end{proof}


\problem{5}
\subproblem{a: $\Z/2\Z \times \Z/2\Z \times \Z/2\Z$}
\begin{proof}
  Let $\K = \Q(\sqrt p, \sqrt q, \sqrt r)$ for $p,q,r$ three distinct prime numbers.
  Then we know the minimal polynomials for these three generators are $x^2 - p$, $x^2 - q$, and $x^2 - r$ respectively, since all three equations are irreducible with the corresponding generator as a root.
  Also this extension is Galois since it is the smallest field containing all the roots of the separable polynomial obtained by multiplying all of these minimal polynomials together (So it is the splitting field for their product which is separable).
  
  Then any automorphism $\sigma \in Aut(\K/\Q)$ is characterized by its action on the three generators, and it must send each generator to one of the roots of the corresponding minimal polynomial, so we can characterize such a $\sigma$ by:
  \begin{multicols}{3}
    \noindent
    \begin{equation*}
      \sigma(\sqrt p) = \pm \sqrt p
    \end{equation*}
    \noindent
    \begin{equation*}
      \sigma(\sqrt q) = \pm \sqrt q
    \end{equation*}
    \noindent
    \begin{equation*}
      \sigma(\sqrt r) = \pm \sqrt r
    \end{equation*}
  \end{multicols}
  Therefore the maps can be represented by triples $(i,j,k)$ with $i,j,k \in \Z/2\Z$, where each triple corresponds to the map negating $\sqrt p$ if $i = 1$ and fixing it if $i = 0$ and so on for $\sqrt q$ with $j$ and $\sqrt r$ with $k$.
  We argue that this representation is actually an isomorphism.
  
  First, composition of these maps corresponds to component-wise addition modulo $2$, since the negations corresponding to each component are independent self-inverses.
  Also, the identity map corresponds to the identity element under component-wise addition, namely $(0,0,0)$.
  Finally we note that this representation covers all of $\Z/2\Z \times \Z/2\Z \times \Z/2\Z$, since any assignment of $0$s and $1$s has a corresponding combination of negations on the generators.
  Therefore this is a surjective homomorphism between finite groups, and we conclude $Aut(\K/\Q) \cong \Z/2\Z \times \Z/2\Z \times \Z/2\Z$.
\end{proof}

\subproblem{b: 16}
\begin{proof}
  By the Fundamental Theorem of Galois Theory it suffices to count the subgroups of $Aut(\K/\Q)$.
  Since this group has order $8$, any non-trivial subgroup has order $2$ or $4$ by Lagrange.

  Any subgroup of order $2$ must necessarily be generated by a single element of order $2$.
  But each element of $Aut(\K/\Q)$ besides the identity has order $2$ by the previous characterization, so this gives $7$ distinct subgroups.

  Then consider any two distinct non-identity elements $\sigma,\tau$ in some subgroup of $Aut(\K/\Q)$.
  No two distinct elements in the group compose to $1$, so then the third non-trivial element must be their product.
  Also $\{1, \sigma, \tau, \sigma\tau\}$ is closed under composition since every element has order $2$ and the group is Abelian.
  Therefore this completely characterizes the subgroups of order $4$.
  This naively gives $7*6$ subgroups by choosing the distinct $\sigma, \tau$.
  But note this actually over counts by a factor of $3! = 6$, since any permuation of $\sigma, \tau, \sigma\tau$ gives the same subgroup.
  So this actually gives exactly $7$ subgroups.

  So in total we get $2$ trivial subgroups, $7$ of order $2$, and $7$ of order $4$, giving a total of $16$ subgroups, and in turn $16$ intermediate field exensions.
\end{proof}

\subproblem{c: All 16}
\begin{proof}
  Since the Galois group is Abelian, every subgroup of $Aut(\K/\Q)$ is normal, and so by the Fundamental Theorem of Galois Theory every subfield of $\K$ is Galois over $\Q$.
\end{proof}

\subproblem{d: 2}
\begin{proof}
  By the Fundamental Theorem of Galois Theory, it suffices to count which subgroups of $Aut(\K/\Q)$ fix $\alpha := 10\sqrt{p}-3/7\sqrt{r}$.
  But by the previous characterization, the only automorphisms that do this are the identity and the isomorphism $\sigma$ negating $\sqrt{q}$ and fixing the other generators.
  These two elements generate only two different subgroups, $\{1\}$ and $\{1,\sigma\}$.
  Therefore we conclude that the only two subfields containing $\alpha$ are the two corresponding fixed fields for these subgroups.
\end{proof}

\end{document}