%&pdflatex
\documentclass[11pt]{article}

\usepackage{main-macros}

\usepackage{newpxmath}
\usepackage{newpxtext}
\usepackage[margin=0.75in]{geometry}

\newcommand{\disc}{\text{disc}}
\newcommand{\legendre}[2]{\genfrac{(}{)}{}{}{#1}{#2}}

\DeclareMathOperator{\Gal}{Gal}

\title{8251 HW2}
\author{Devon Tuma}
\date{Fall 2020}

\begin{document}
\maketitle

\problem{4.4}

\subproblem{a}
Prove that $\legendre{aa'}{b} = \legendre{a}{b}\legendre{a'}{b}$, and that $\legendre{a}{bb'} = \legendre{a}{b}\legendre{a}{b'}$.
\begin{proof}
  Let $b = \prod_i p_i^{r_i}$ be a prime factorization of $b$, and assume $(a,b) = (a',b) = 1$. Then by considering prime factorizations $a$ and $a'$ are not divisible by any of the $p_i$. Therefore we can use the result of 4.3.(a) to get:
  \begin{equation*}
    \legendre{aa'}{b} = \legendre{aa'}{\prod_i p_i^{r_i}}
    = \prod_i \legendre{aa'}{p_i}^{r_i}
    = \prod_i \legendre{a}{p_i}^{r_i} \legendre{a'}{p_i}^{r_i}
    = \prod_i \legendre{a}{p_i}^{r_i} \prod_i \legendre{a'}{p_i}^{r_i}
    = \legendre{a}{\prod_i p_i^{r_i}} \legendre{a'}{\prod_i p_i^{r_i}}
    = \legendre{a}{b} \legendre{a'}{b}
  \end{equation*}
  If we then let $b' = \prod_j q_j^{s_j}$ be a prime factorization of $b'$, and further assume $(a,b') = 1$, then using the definitions we get:
  \begin{equation*}
    \legendre{a}{bb'} = \legendre{a}{\prod_i p_i^{r_i} \prod_j q_j^{s_j}}
    = \prod_i \legendre{a}{p_i}^{r_i} \prod_j \legendre{a}{q_j}^{s_j}
    = \legendre{a}{\prod_i p_i^{r_i}} \legendre {a}{\prod_j q_j^{s_j}}
    = \legendre{a}{b} \legendre{a}{b'}
  \end{equation*}
\end{proof}

\subproblem{b}
Prove that if $\equivmod{a}{a'}{b}$ then $\legendre{a}{b} = \legendre{a'}{b}$.
\begin{proof}
  Since $a \equiv a' \pmod b$, we also have $a \equiv a' \pmod p_i$ for each prime factor $p_i$ of $b$. Therefore $\legendre{a}{p_i} = \legendre{a'}{p_i}$ for each prime factor of $b$ since the Legendre symbol is defined by a property of the reduction modulo $p_i$. Therefore we get
  \begin{equation*}
    \legendre{a}{b} = \legendre{a}{\prod_i p_i^{r_i}}
    = \prod_i \legendre{a}{p_i}^{r_i}
    = \prod_i \legendre{a'}{p_i}^{r_i}
    = \legendre{a'}{\prod_i p_i^{r_i}}
    = \legendre{a'}{b}
  \end{equation*}
\end{proof}

\subproblem{c}
Prove that $\legendre{-1}{b}$ is $1$ if $\equivmod{b}{1}{4}$ and is $-1$ if $\equivmod{b}{-1}{4}$.
\begin{proof}
  We already saw that this identity holds in the special case where $b$ is a prime and this is the Legendre symbol (problem 4.3.(a)).
  Hence by the multiplicative definition of the Jacobi symbol, $\legendre{-1}{b} = 1$ if and only if an even number of prime factors of $b$ (with multiplicity) are congruent to $-1$ modulo $4$.
  But $b$ having an even number of factors being congruent to $-1$ is exactly equivalent to $b$ itself being equivalent to $1$ modulo $4$.
  So we conclude $\legendre{-1}{b} = 1$ if $b \equiv 1 \pmod 4$ and is $-1$ otherwise, as desired.
\end{proof}

\subproblem{d}
Prove that $\legendre{2}{b}$ is $1$ if $\equivmod{b}{\pm 1}{8}$ and is $-1$ if $\equivmod{b}{\pm 3}{8}$.
\begin{proof}
  Again we know that this formula holds in the special case where $b$ is a prime and this is the Legendre symbol by the quadratic reciprocity law, so by the multiplicative definition of the Jacobi symbol, $\legendre{2}{b} = 1$ if an even number of prime factors of $b$ (with multiplicity) are congruent to $\pm 3$ modulo $8$.
  But note that $3^2 \equiv (-3)^2 \equiv 1 \pmod 8$ and $-3 * 3 \equiv 3 * -3 \equiv -1 \pmod 8$, so having an even number of factors being congruent to $\pm 3$ is exactly equivalent to $b$ itself being equivalent to $\pm 1$ module $8$.
  So we conclude $\legendre{2}{b} = 1$ if $b \equiv \pm 1 \pmod 8$ and is $-1$ otherwise, as desired.
\end{proof}

\subproblem{e}
Prove that $\legendre{a}{b}$ is $\legendre{b}{a}$ if $\equivmod{a \text{ or } b}{1}{4}$ and is $-legendre{b}{a}$ otherwise.
\begin{proof}
  Again, assume $(a,b) = 1$ so that $\legendre{a}{b}$ is well defined. Then using part (a) we get the following identity (up to a yet to be determined sign):
  \begin{equation*}
    \legendre{a}{b} = \legendre{a}{\prod_i p_i^{r_i}}
    = \prod_i \legendre{a}{p_i}^{r_i}
    = \stackrel{?}{\pm} \prod_i \legendre{p_i}{a}^{r_i}
    = \stackrel{?}{\pm} \legendre{\prod_i p_i^{r_i}}{a}
    = \stackrel{?}{\pm} \legendre{b}{a}
  \end{equation*}
  where the final sign is $-1$ if and only if $a \equiv -1 \pmod 4$ and an odd number of the $p_i$ (with multiplicity) are equivalent to $-1 \pmod 4$.
  But having an odd number of factors congruent to $-1$ is equivalent to $b$ being congruent to $-1$ itself, so we conclude that the final sign is $-1$ iff $a \equiv b \equiv -1 \pmod 4$, exactly as desired. 
\end{proof}

\subproblem{f}
Use the Jacobi symbol to calculate $\legendre{2413}{4903}$.
\begin{proof}
  using the previously proved identities:
  \begin{align*}
    \legendre{2413}{4903}
    = \legendre{4903}{2413} \text{ by (e)} 
    &= \legendre{77}{2413} \text{ by (b)} 
    = \legendre{26}{77} \text{ by (e) and (b)}
    = \legendre{2}{77} \legendre{13}{77} \text{ by (a)} 
    = - \legendre{13}{77} \text{ by (d)} \\
    &= - \legendre{12}{13} \text{ by (e) and (b)} 
    = - \legendre{3}{13} \text{ by (a) and (d)} 
    = - \legendre{1}{3} \text{ by (e) and (b)} 
    = -1
  \end{align*}
\end{proof}


\problem{4.10}
Let $K$ be a number field, and let $L$ and $M$ be two finite extensions of $K$.
Assume that $M$ is normal over $K$.
Then the composite field $LM$ is normal over $L$ and the Galois group $Gal(LM/L)$ is embedded in $Gal(M/K)$ by restricting automorphisms to $M$.
Let $P, Q, U, V$ be primes in $K, L, M, LM$ respectively such that $V$ lies over $Q$ and $U$, and $Q$ and $U$ lie over $P$.
\subproblem{a}
Prove that $D(V|Q)$ is embedded in $D(U|P)$ by restricting automorphisms
\begin{proof}
  Let $\sigma \in D(V|Q)$, so that $\sigma \in Gal(LM/L)$ and $\sigma(V) = V$, and we want  to show that $\sigma$ embeds into $D(U|P)$ by restricting to $\sigma|_M$.
  We know that $\sigma|_M \in Gal(M/K)$, so it suffices to check that $\sigma|_M(U) = U$.
  Given some $u \in U$, then we also have $u \in V$ since $V$ lies over $U$, and so by assumption $\sigma(u) \in V$.
  But then $\sigma|_M(u) = \sigma(u)$, and so $\sigma(u)$ must be an element of $M$, and so lies in $V \cap M = U$, and we conclude $\sigma(U) \le U$.
  On the other hand if $\sigma|_M(u) \in \sigma|_M(U)$, then $\sigma(u) \in \sigma(V)$ by the fact $\sigma \in D(V|Q)$, and so $u \in V$.
  Hence since $u \in M$ by assumption we get $u \in V \cap M = U$, and hence $U \le \sigma(U)$, completing the proof.
\end{proof}

\subproblem{b}
Prove that $E(V|Q)$ is embedded in $E(U|P)$ be restricting automorphisms.
\begin{proof}
  Using the same reasoning as in (a), it suffices to show that if $\equivmod{\sigma(\alpha)}{\alpha}{V}$ for all $\alpha \in LM$, then $\equivmod{\sigma|_M(\alpha)}{\alpha}{U}$ for all $\alpha \in M$.
  Given any $\alpha \in M$, we can embed $\alpha \in LM$, and so by assumption we have $\equivmod{\sigma(\alpha)}{\alpha}{V}$, so that $\sigma(\alpha) - \alpha \in V$.
  But then $\sigma(\alpha) - \alpha$ is also in $M$, since we assumed $\alpha \in M$ and $\sigma$ fixes $L \subset LM$.
  So $\sigma|_M(\alpha) - \alpha \in V \cap M = U$, and we get that $\equivmod{\sigma|_M(\alpha)}{\alpha}{U}$ as desired.
\end{proof}

\subproblem{c}
Prove that if $P$ is unramified in $M$ then every prime of $L$ lying over $P$ is unramified in $LM$.
\begin{proof}
  We prove the contrapositive, so assume some prime $Q$ lying over $P$ is ramified in $LM$, and we need to show $P$ is ramified in $M$.
  Since $Q$ is ramified, we get that $LM/(LM)_E$ is a nontrivial extension, and so $E(R|Q)$ is nontrivial for any $R$ lying over $Q$.
  Then by part (b) we can embed this inertia group into $E(Q'|P)$ for any prime $Q'$ lying over $P$, and so the lower inertia group is also nontrivial.
  Finally then we get that $M/M_E$ is a nontrivial extension, and so $P$ is ramified in $M$.
\end{proof}

\subproblem{d}
Prove that if $P$ splits completely in $M$ then every prime of $L$ lying over $P$ splits completely in $LM$.
\begin{proof}
  The same argument as in (c) applied using the decomposition groups and part (a) instead of the inertia groups and part (b) gives the desired result.
\end{proof}

\subproblem{e}
Assume that $P$ is unramified in $M$, so that $\phi(U|P)$ and $\phi(V|Q)$ are defined. Prove that the restriction of $\phi(V|Q)$ to $M$ is $\phi(U|P)^{f(Q|P)}$.
\begin{proof}
  First, it suffices to show that these both induce the same automorphism of $(\mathbb{A} \cap M)/U$, since then using (a) they must agree on ideals of $LM$ as well, and so by uniqueness of the frobenius automorphism will be the same.
  So let some $\alpha \in (\mathbb{A} \cap M)$ be given, and note that $\equivmod{\phi(V|Q)(\alpha)}{\alpha^{||Q||}}{V}$ by definition of the frobenius automorphism.
  On the other hand $\equivmod{\phi(U|P)^{f(Q|P)}(\alpha)}{\alpha^{f(Q|P)||P||}}{U}$, and so pulling both back down to $M$ using (b), and the fact $Q$ lies over $P$, we see that both induce the value $U$, which by the uniqueness of the frobenius automorphism implies that they in fact agree on all of $M$, completing the proof.
\end{proof}


\problem{4.11}
Consider the special case of the setup in 4.10 with $K \subset L \subset M$, with $M$ normal over $K$ and $P$ unramified in $M$.
\subproblem{a}
Show that $\phi(U|Q) = \phi(U|P)^{f(Q|P)}$.
\begin{proof}
  Since $L \subset M$ we have $LM = M$, since $M$ already contains $L$.
  Also we have $V = U$, since $V$ lies over $U$ in the trivial extension.
  Plugging these into the result of 4.10.e gives that $\phi(U|Q)$ restricts to $\phi(U|P)^{f(Q|P)}$, but this restriction is trivial since the extension is trivial, so in fact the two maps must be equal, giving the desired equation.
\end{proof}

\subproblem{b}
Suppose $L$ is also normal over $K$, so $\phi(Q|P)$ is defined.
Prove that $\phi(Q|P)$ is the restriction of $\phi(U|P)$ to $L$.
\begin{proof}
  By uniqueness of the frobenius automorphism, it suffices to check that the restriction of $\phi(U|P)$ to $L$ satisfies the defining equivalence of $\phi(Q|P)$.
  Therefore we need to check that $\equivmod{\phi(U|P)(\alpha)}{\alpha^{||P||}}{Q}$ for all $\alpha \in L$.
  Equivalently, it suffices to check that $\equivmod{\phi(U|P)(\alpha)^{f(Q|P)}}{\alpha^{f(Q|P)||P||}}{Q}$, and so using the result of (a) it suffices to show $\equivmod{\phi(U|Q)(\alpha)}{\alpha^{f(Q|P)||P||}}{Q}$.
  But then since $||Q|| = f(Q|P)||P||$ and $\alpha \in L \subset M$, this is just the defining equivalence for the frobenius map $\phi(U|Q)$, and so certainly holds true, completing the proof.
\end{proof}


\problem{4.15}
Let $F$ be any field. Show that distinct automorphisms $\sigma_1, \dots \sigma_n$ of $F$ are always linearly independent over $F$.

\begin{proof}
  Assume for contradiction we have a linear dependence $a_1\sigma_1 + \dots a_n\sigma_n = 0$, with $a_i \in F$ not zero, and further assume $n$ is minimal such that this happens.
  Since the $\sigma_i$ are distinct, there exists some $x \in F$ with $\sigma_1(x) \ne \sigma_n(x)$.
  Then we have point-wise equality between $\sum_i a_i \sigma_i(x) \sigma_i$ and the zero automorphism, since for every $y \in F$ we have:
  \begin{equation*}
    \parens*{\sum_i a_i \sigma_i(x) \sigma_i}(y)
    = \sum_i a_i \sigma_i(x) \sigma_i(y)
    = \sum_i a_i \sigma_i(xy)
    = \parens*{\sum_i a_i \sigma_i}(xy)
  \end{equation*}
  And so we have $\sum_i a_i \sigma_i(x) \sigma_i = 0$.
  But we also have $\sum_i a_i \sigma_1(x) \sigma_i = \sigma_1(x) \sum_i a_i \sigma_i = 0$, and subtracting these two gives $\sum_{i \ne 1} a_i (\sigma_i - \sigma_1)(x) \sigma_i = 0$.
  But then note that $(\sigma_n - \sigma_1)(x) \ne 0$ by our choice of $x$ so this is a new nontrivial linear dependence, so this contradicts the assumption that $n$ is minimal.
  Therefore we conclude that the $\sigma_i$ are in fact linearly independent.
\end{proof}


\problem{4.29}
Show that if $K$ and $L$ are both abelian extensions of $\Q$ then so is $KL$.

\begin{proof}
  It suffices to show that the galois group $\Gal(KL/\Q)$ can be embedded in the direct product $\Gal(K/\Q) \times \Gal(L/\Q)$, since the direct product of Abelian groups is Abelian.
  Since both $K$ and $L$ are contained in the compositum $KL$, we can define a map $\varphi : Gal(KL/\Q) \rightarrow \Gal(K/\Q) \times \Gal(L/\Q)$ by restriction $\sigma \mapsto (\sigma|_K, \sigma|_L)$.
  This is clearly a group homomorphism, since the sum of restrictions is the restriction of sums, and the identity map restricts to the identity map, so we need only check that this map is injective.

  But every element of $KL$ can be expressed as combination of sums, products, and inverses of elements in $K$ and $L$, so any automorphism that is the identity on both $K$ and $L$ is also the identity on $KL$.
  Therefore if both the restrictions $\sigma|_K$ and $\sigma|_L$ are the identity, we also have that $\sigma$ is the identity.
  Hence $\ker \varphi$ is trivial, and $\varphi$ is an embedding, completing the proof.
\end{proof}


\problem{4.30}
Show that every abelian extension of $\Q$ is the composition of abelian extensions of prime power degree.

\begin{proof}
  Let $K$ be an abelian extension of $\Q$ and $G = \Gal(K/\Q)$ its Galois group.
  Since $K$ is an abelian extension, $G$ is a finite Abelian group, and we can write $G = \prod_i (\Z/p_i\Z)^{r_i}$ for some primes $p_i$ by the fundamental theorem of finite Abelian groups.
  Then consider the following sequence of inclusions of Abelian groups:
  \begin{equation*}
    0 \xhookrightarrow{} (\Z/p_1\Z)^{r_1}
    \xhookrightarrow{} (\Z/p_1\Z)^{r_1} \times (\Z/p_2\Z)^{r_2}
    \xhookrightarrow{} \dots \xhookrightarrow{}
    \prod_i (\Z/p_i\Z)^{r_i} = G    
  \end{equation*}
  By the fundamental theorem of Galois theory this corresponds to a sequence of field extensions:
  \begin{equation*}
    \Q \rightarrow{} K_1 \rightarrow{} K_2 \rightarrow{} \dots
    \rightarrow{} K
  \end{equation*}
  where the degree of each extension is exactly the index of the corresponding Galois group.
  But each of the extensions of the Galois groups was of prime power index by construction, so all these extensions are of prime power degree as well, and so we are done.
\end{proof} 


\problem{5.4}
Prove that the subset of $\R^n$ defined by the inequalities
\begin{equation*}
  |x_1| + \dots + |x_r| + 2\parens{\sqrt{x_{r+1}^2 + x_{r+2}^2} + \dots + \sqrt{x_{n-1}^2 + x_{n}^2}} \le n
\end{equation*}
is convex.

\begin{proof}
  First, note that this inequality defines a closed set, since the inequality is non-strict (and hence the limit of a sequence of points satisfying the inequality will also satisfy it). Then given and line between points in the set, and a point on that line, note that the point is the limit of the recursive sequence of midpoints on the line approaching the given point. Therefore since the set is closed, it suffices to show that all the midpoints in this sequence are in the set, and then the limit will also be in the set.

  It then remains to show that if $(x_1, \dots, x_n)$ and $(y_1, \dots, y_n)$ are points satisfying the given inequality, then so does $\frac{1}{2}(x_1 + y_1, \dots, x_n + y_n)$. By assumption we have:
  \begin{equation*}
    |x_1| + \dots + |x_r| + 2\parens*{\sqrt{x_{r+1}^2 + x_{r+2}^2} + \dots + \sqrt{x_{n-1}^2 + x_{n}^2}} \le n
  \end{equation*}
  \begin{equation*}
    |y_1| + \dots + |y_r| + 2\parens*{\sqrt{y_{r+1}^2 + y_{r+2}^2} + \dots + \sqrt{y_{n-1}^2 + y_{n}^2}} \le n
  \end{equation*}
  Then using the triangle inequalities in $\R$ and $\R^2$ we calculate:
  \begin{align*}
    |x_1 + y_n| &+ \dots + |x_r + y_r| + 2\parens*{\sqrt{(x_{r+1} + y_{r+1})^2 + (x_{r+2} + y_{r+2})^2} + \dots + \sqrt{(x_{n-1} + y_{n-1})^2 + (x_{n} + y_{n})^2}} \\
                &\le |x_1| + |y_n| + \dots + |x_r| + |y_r| + 2\parens*{\sqrt{(x_{r+1} + y_{r+1})^2 + (x_{r+2} + y_{r+2})^2} + \dots + \sqrt{(x_{n-1} + y_{n-1})^2 + (x_{n} + y_{n})^2}} \\
                &\le |x_1| + |y_n| + \dots + |x_r| + |y_r| + 2\parens*{\sqrt{x_{r+1}^2 + x_{r+2}^2} + \sqrt{y_{r+1}^2 + y_{r+2}^2} + \sqrt{x_{n-1}^2 + x_n^2} + \sqrt{y_{n-1}^2 + y_n^2}} \\
                &\le |x_1| + \dots + |x_r| + 2\parens*{\sqrt{x_{r+1}^2 + x_{r+2}^2} + \dots + \sqrt{x_{n-1}^2 + x_{n}^2}} + \\ &\quad \quad \quad \quad |y_1| + \dots + |y_r| + 2\parens*{\sqrt{y_{r+1}^2 + y_{r+2}^2} + \dots + \sqrt{y_{n-1}^2 + y_{n}^2}} \\
                &\le n + n = 2n
  \end{align*}
  And so we conclude $(x_1 + y_1, \dots, x_r + y_r) \le 2n$, which is exactly what we needed to show.
\end{proof}


\problem{5.5}
Prove by induction that $\frac{n^n}{n!} \ge 2^{n-1}$ and use this to show that $|disc(R)| \ge 4^{r-1}\pi^{2s}$.
\begin{proof}
  We proceed by induction on $n$. When $n = 1$ we have $\frac{1^1}{1!} = 1 = 2^{1-1}$. Then assume the formula holds for $k$ and we calculate:
  \begin{equation*}
    \frac{(n+1)^{n+1}}{(n+1)!} = \frac{(n+1)^n}{n!} \ge \frac{2n^n}{n!} \ge 2*2^{n-1} = 2^n
  \end{equation*}
  Where the third inequality is the induction hypothesis, and the second inequality follows from the inequality $(n+1)^n \ge (2^{1/n}n)^n = 2n^n$, which is true since $1 + \frac{1}{n} \ge 2^{1/n}$ for any $n \ge 1$. Note that $1 + \frac{1}{n} \ge 2^{1/n}$ follows by just comparing the partial differences $\frac{1}{n} - \frac{1}{n+1} \ge \frac{1}{2^n}$ and checking a base case at $n=1$. Therefore we conclude $\frac{n^n}{n!} \ge 2^{n-1}$ by induction as desired.

  Then using theorem 37 we also have the other desired inequality:
  \begin{equation*}
    |disc(R)| \ge \parens{\frac{n^n}{n!}\parens*{\frac{\pi}{4}}^s}^2 \ge \parens{2^{n-1}4^{-s}\pi^s}^2 = \parens{2^{r-1}\pi^s}^2 = 4^{r-1}\pi^{2s}
  \end{equation*}
\end{proof}


\problem{5.7}
Show that there are two ideal classes in $\Z[\sqrt{10}]$.
\begin{proof}
  Taking $\{1,\sqrt{10}\}$ as a basis for $\Z[\sqrt{10}]$, theorem 35 implies each ideal class has an ideal of norm $||J|| \le (1 + \sqrt{10})^2 \le 18$. Using theorem 37, we can refine this to get $||J|| \le 4\sqrt{10}/\pi < 5$. Therefore we consider primes lying over $2$ and $3$, and using theorem 25 we find:
  $$2R = (2, \sqrt{10})^2 \quad \text{since}\ 2 | 10$$
  $$3R = (3, 1+\sqrt{10})(3, 1-\sqrt{10}) \quad \text{since}\ \equivmod{10}{1^2}{3} $$
  None of these three ideals on the rhs are principle, since for any $a + b\sqrt{10} \in \Z[\sqrt{10}]$ we have $N(a + b\sqrt{10}) = a^2 - b^2m$, which can't be $2$ or $3$ since they aren't squares modulo $m = 10$.

  Then consider the element $\alpha = 6 + 3\sqrt{10}$ which has norm $N(6 + 3\sqrt{10}) = 36 - 30 = 6$.
  Since it has norm $6$, we know the principle ideal $(\alpha)$ splits into an ideal over $2$ and an ideal over $3$.
  But this implies that the class containing $(2, \sqrt{10})$ is the inverse class of the class containing either of the $(3, 1 \pm \sqrt{10})$. But also we have that the class of $(2, \sqrt{10})$ is order $2$ by the equation for $2R$, and so is its own inverse as well.
  Therefore since inverses are unique we conclude that $(2, \sqrt{10})$ and $(3, 1 \pm \sqrt{10})$ all lie in the same nontrivial ideal class, and conclude that there are in total $2$ ideal classes.
\end{proof}


\problem{5.8}
Show that $\Z[\sqrt{233}]$ has three ideal classes.
\begin{proof}
  Taking $\{1,\sqrt{223}\}$ as a basis for $\Z[\sqrt{223}]$, theorem 35 implies each ideal class has an ideal of norm $||J|| \le (1 + \sqrt{223})^2 \le 254$. Using theorem 37, we can refine this to get $||J|| \le 4\sqrt{223}/\pi < 20$. Therefore we consider primes lying over $2,3,5,7,11,13,17,19$, and using theorem 25 we find:
  \begin{align*}
    2R &= (2, 1+\sqrt{223})(2, 1-\sqrt{223}) \\
    3R &= (3, 1+\sqrt{223})(3, 1-\sqrt{223}) \\
    5R &= \text{prime} \\
    7R &= \text{prime} \\
    11R &= (11, 5+\sqrt{223})(11, 5-\sqrt{223}) \\
    13R &= (13, 9+\sqrt{223})(13, 6-\sqrt{223}) \\
    17R &= (17, 6+\sqrt{223})(17, 9-\sqrt{223}) \\
    19R &= \text{prime} \\
  \end{align*}
  The norm of any single element $a + b\sqrt{223}$ is $a^2 - 223b^2$, and so the norm is always a square mod $223$.
  Using the Jacobi symbol, we can check that $3$, $11$, and $13$ are not squares mod $223$, so none of the ideals they split into are principle.
  On the other hand, $2$ is a square mod $223$, and both of the ideals it splits into are equal to the principle ideal $(15 - \sqrt{223})$ (since $-(15 - \sqrt{223})^2 + 2(15 - \sqrt{223}) = 2$ all the generators can be constructed as linear combinations of the one generator).
  Similarly $17$ is a square mod $223$ and both ideals it splits into are equal to the principle ideal $(54 - \sqrt{223})$ by a similar construction.
  Therefore it remains to classify the ideal classes containing the primes lying over $3$, $11$, and $13$.
  
  Consider the element $\alpha = 16 + \sqrt{223}$ which has norm $N(16 + \sqrt{223}) = 16^2 - 223 = 33$.
  Therefore the principle ideal $(\alpha)$ splits into a prime lying over $3$ and a prime lying over $11$.
  Therefore the class containing this ideal over $3$ is inverse to the class containing the ideal over $11$.
  But we also know that the ideals over $3$ and $11$ are inverses with each other, so by the uniqueness of inverses, each of the ideals over $3$ is in the same class as one of the ideals over $11$.
  Furthermore, all $4$ can't lie in the same ideal class, since the opposite cross products of the classes containing the ideals are not principle ideals.
  Therefore these $4$ ideals lie in two distinct non-trivial ideal classes.
  A similar argument shows that the same is true for the ideals over $3$ and $13$, so we conclude that in total there are 2 non-trivial ideal classes, and hence $\Z[\sqrt{223}]$ has 3 ideal classes.
\end{proof}


\problem{5.33}
\subproblem{a}
Let $m$ be a square-free positive integer, and assume that $\equivmod{m}{2 \text{ or } 3}{4}$.
Let $b$ be minimal such that $mb^2+1$ or $mb^2-1$ is a square, and assume it is $a^2$ with $a > 0$.
Prove that $a + b\sqrt{m}$ is the fundamental unit.

\begin{proof}
  First, $a+b\sqrt{m}$ is a unit since it divides a unit:
  \begin{equation*}
    (a+b\sqrt{m})(a-b\sqrt{m})
    = a^2 - b^2m = mb^2 \pm 1 - mb^2 = \pm 1
  \end{equation*}
  Then assume that the fundamental unit is $c + d\sqrt{m}$ with $c,d \in \Z^+$.
  Units have norm $1$, so we have $N(c+d\sqrt{m}) = c^2 - md^2 = 1$ and so $md^2 + 1$ is a square.
  Hence by minimality of $b$, we have $b \le d$.
  Then we also have $c^2 = d^2m + 1 \ge bm^2 + 1 = a^2$, and so $c \ge a$ as well, giving $c + d\sqrt{m} \ge a + b\sqrt{m}$.
  But we assumed that $c + d\sqrt{m}$ was the fundamental unit, so $(c+d\sqrt{m})^k = a + b\sqrt{m}$ for some $k \ge 1$, and so $c + d\sqrt{m} \le a + b\sqrt{m}$ as well.
  Therefore we conclude $c + d\sqrt{m} = a + b\sqrt{m}$, and $a + b\sqrt{m}$ is the fundamental unit, which is what we wanted to show.
\end{proof}

\subproblem{b}
Establish a similar procedure for determining the fundamental unit in $\mathbb{A} \cap \Q[\sqrt{m}]$ for square-free $m > 1$, $\equivmod{m}{1}{4}$.

\begin{proof}
  Take $b$ to be minimal such that $mb^2+4$ or $mb^2-4$ is a square, and assume it is $a^2$ for some $a \ge 0$.
  We argue that $(a + b\sqrt{m})/2$ is the fundamental unit.
  First note that it is a unit since it divides a unit:
  \begin{equation*}
    (a+b\sqrt{m})(a-b\sqrt{m})/4
    = (a^2 - b^2m)/4 = (mb^2 \pm 4 - mb^2)/4 = \pm 1
  \end{equation*}
  Then assume that the fundamental unit is $(c + d\sqrt{m})/2$ with $c,d \in \Z^+$.
  Units have norm $1$, so we have $N((c+d\sqrt{m})/2) = (c^2 - md^2)/4 = 1$ and so $md^2 + 4$ is a square.
  Hence by minimality of $b$, we have $b \le d$.
  Then we also have $c^2 = d^2m + 4 \ge bm^2 + 4 = a^2$, and so $c \ge a$ as well, giving $(c + d\sqrt{m})/2 \ge (a + b\sqrt{m})/2$.
  But we assumed that $(c + d\sqrt{m})/2$ was the fundamental unit, so $((c+d\sqrt{m})/2)^k = (a + b\sqrt{m})/2$ for some $k \ge 1$, and so $(c + d\sqrt{m})/2 \le (a + b\sqrt{m})/2$ as well.
  Therefore we conclude $(c + d\sqrt{m})/2 = (a + b\sqrt{m})/2$, and $(a + b\sqrt{m})/2$ is the fundamental unit, which is what we wanted to show.
\end{proof}


\problem{5.34}
The following python code implements the algorithm from the previous problem
\begin{verbatim}
from math import sqrt
def is_square(n):
    if int(sqrt(n))**2 == n:
        return int(sqrt(n))
    return None

def calc_unit(m):
    if m % 4 == 0:
        return -1
    if m % 4 == 3 or m % 4 == 2: k = 1
    else: k = 4
    a, b = None, 0
    while a == None:
        b += 1
        a = is_square(m * b * b + k)
    return "{} + {}*sqrt({})".format(a, b, m)
\end{verbatim}
Which gives the following results
\begin{verbatim}
2 -> 3 + 2*sqrt(2)
3 -> 2 + 1*sqrt(3)
5 -> 3 + 1*sqrt(5)
6 -> 5 + 2*sqrt(6)
7 -> 8 + 3*sqrt(7)
10 -> 19 + 6*sqrt(10)
11 -> 10 + 3*sqrt(11)
13 -> 11 + 3*sqrt(13)
14 -> 15 + 4*sqrt(14)
15 -> 4 + 1*sqrt(15)
17 -> 66 + 16*sqrt(17)
19 -> 170 + 39*sqrt(19)
21 -> 5 + 1*sqrt(21)
22 -> 197 + 42*sqrt(22)
23 -> 24 + 5*sqrt(23)
26 -> 51 + 10*sqrt(26)
29 -> 27 + 5*sqrt(29)
30 -> 11 + 2*sqrt(30)
\end{verbatim}


\problem{6.4}
Let $K$ be a number field, $R = \mathbb{A} \cap K$.
Define a relation $\overset{+}{\sim}$ on the nonzero ideals of $R$ as $I \overset{+}{\sim} J$ iff $\alpha I = \beta J$ for some $\alpha, \beta \in R^+$.

\subproblem{a}
Prove this is an equivalence relation
\begin{proof}
  Reflexivity is clear, since clearly $1 \in R^+$ and so $1I = 1I$ for all $I$.
  Similarly if $I \overset{+}{\sim} J$ then $\alpha I = \beta J$ for some $\alpha, \beta$, and so $\beta J = \alpha I$ and hence $J \overset{+}{\sim} I$ and the relation is symmetric.

  For transitivity, first not that the product of totally positive elements is totally positive, since if $\sigma(\alpha) > 0$ and $\sigma(\beta) > 0$ then $\sigma(\alpha\beta) = \sigma(\alpha)\sigma(\beta) > 0$ as well.
  Then if we assume $I \overset{+}{\sim} J$ and $J \overset{+}{\sim} K$, we will have $\alpha I = \beta J$ and $\gamma J = \delta K$ for some $\alpha,\beta,\gamma,\delta \in R^+$.
  But then $\gamma \alpha I = \gamma \beta J = \beta \gamma J = \beta \delta K$, where $\gamma \alpha$ and $\beta \delta$ are in $R^+$ by the multiplicative closure property.
  So we conclude $I \overset{+}{\sim} K$, and therefore the relation is transitive and hence an equivalence relation.
\end{proof}

\subproblem{b}
Prove that the equivalence classes under this relation form a group $G^+$ in which the identity element is the class consisting of all principle ideals $(\alpha)$, $\alpha \in R^+$.
\begin{proof}
  We take the same group operation as the ideal class group, so $\bar I \bar J = \bar{IJ}$ for some representatives $I$ and $J$.
  This is well defined since if $I \overset{+}{\sim} I'$ and $J \overset{+}{\sim} J'$, with $\alpha I = \alpha' I'$ and $\beta J = \beta' J'$ for $\alpha, \alpha', \beta, \beta' \in R^+$, then we have $\alpha \beta I J = \alpha' \beta I' J = \alpha' \beta' I' J'$ and so $IJ \overset{+}{\sim} I'J'$.

  The class of principle ideals $(\alpha)$ with $\alpha \in R^+$ is an identity for this operation, $I(\alpha) = \alpha I$ for any ideal $I$, which implies $I(\alpha) \overset{+}{\sim} I$ and hence $\bar I \bar{(\alpha)} = \bar I$, making this class a left identity.
  A similar argument shows that this is a right identity as well.
  The operation is also associative, since $(IJ)K \overset{+}{\sim} I(JK)$ follows from reflexivity and associativity of regular ideal multiplication.

  Finally, we show that this operation has inverses, so let $I$ be a representative of some ideal class.
  Then let $J$ be a representative for the inverse of the class containing $I$ in the regular ideal class group.
  Then $IJ \sim (1)$, so there exists some $\alpha, \beta \in R$ (not necessarily in $R^+$) such that $\alpha IJ \sim \beta (1)$.
  Then note that $\alpha^2, \beta^2 \in R^+$, since $\sigma(x^2) = \sigma(x)^2 > 0$ for any embedding $\sigma$.
  Then we have $\alpha^2 I (\alpha\beta J) = \alpha^2\beta^2(\alpha IJ) = \alpha^2\beta^2(1)$, with $\alpha^2, \beta^2 \in R^+$, and so $I(\alpha\beta J) \overset{+}{\sim} (1)$, and the class of $\alpha \beta J$ is an inverse for the class containing $I$.
\end{proof}

\subproblem{c}
Show that there is a group homomorphisms $f : G^+ \rightarrow G$.
\begin{proof}
  Define $f(\bar J) = \bar J$, so that an ideal class is sent to the ideal class of one of its representatives.
  This is well-defined since if $I \overset{+}{\sim} J$ via $\alpha I = \beta J$ with $\alpha, \beta \in R^+$, then also $\alpha, \beta \in R$ and so $I \sim J$ as well, finally giving $f(\bar I) = f(\bar J)$.

  This respects the identity element, since $f(1_{G^+}) = f(\bar{(\alpha)}) = \bar{(\alpha)} = 1_G$.
  It is also multiplicative, since $f(\bar I \bar J) = f(\bar{IJ}) = \bar{IJ} = \bar I \bar J$.
  Finally, it respects inverses, since $f(\bar I)f((\bar I)^{-1}) = f(\bar I (\bar I)^{-1}) = f(\bar{(1)}) = \bar{(1)}$.
\end{proof}

\subproblem{d}
Prove that the kernel of $f$ has at most $2^r$ elements, where $r$ is the number of real embeddings $\sigma : K \rightarrow \R$.
Conclude that $G^+$ is finite.
\begin{proof}
  First note that if $I$ represents an ideal class in the kernel of $f$, then $I$ must be principle since the identity class in $G$ consists of principle ideals. So it suffices to show that at most $2^r$ classes consist of principle ideals.
  In fact, we will show that if $sgn(\sigma(\alpha)) = sgn(\sigma(\beta))$ for all real embeddings $\sigma$, then $(\alpha)$ and $(\beta)$ represent the same class, bounding the number of ideals by the $2^r$ sign combinations of the $r$ functions.
  This also suffices to show that $G^+$ is finite by the first isomorphism theorem and the fact that $G$ is finite.

  So assume that $sgn(\sigma(\alpha)) = sgn(\sigma(\beta))$ for some $\alpha, \beta \in R$ and all $\sigma : K \rightarrow \R$.
  Then $\alpha\beta \in R^+$, since $\sigma(\alpha)\sigma(\beta) > 0$ since they have the same sign.
  Hence $(\alpha^2)(\alpha\beta)(\alpha) = (\alpha^2)(\alpha^2)(\beta)$, where $\alpha^2, \alpha\beta \in R^+$, and so $(\alpha) \overset{+}{\sim} (\beta)$, which completes the proof.
\end{proof}


\problem{6.9}
Let $u$ be the fundamental unit in a real quadratic field $K$.
\subproblem{a}
Show that $u$ is totally positive iff $N(u) = 1$.
\begin{proof}
  If $K$ is real quadratic, we have exactly one non-trivial real embedding, namely the one fixing $\Q$ and sending $\sqrt{m} \mapsto -\sqrt{m}$.
  Then if $u$ is totally positive, and is of the form $a + b\sqrt{m}$ (the $\equivmod{m}{2 \text{ or } 3}{4}$ case), we have $a \pm b\sqrt{m} > 0$, and so $a > b \sqrt{m}$, and hence $N(u) = a^2 - mb^2 > a^2 - a^2 = 0$, which implies that the norm is $1$ (it must be either $1$ or $-1$ since $u$ is a unit).
  The $\equivmod{m}{1}{4}$ case is the same, but with the additional factors of $1/2$.

  On the other hand, if $N(u) = 1$, then $a^2 - b^2m > 0$, and so $a > b\sqrt{m}$, which means that for either of the real embeddings, $\sigma(u) = \sigma(a + b\sqrt{m}) = a \pm b\sqrt{m} > a \pm a \ge 0$, and so $u$ is totally positive. In the $\equivmod{m}{1}{4}$ case we again get the same with the additional factors of $1/2$ which don't affect the signs, and so again $u$ is totally positive.
\end{proof}

\subproblem{b}
Prove that $h^+ = 2h$ is $u$ is totally positive, and otherwise $h^+ = h$.
\begin{proof}
  We saw in the proof of $6.4.(d)$ that $G^+$ is larger that $G$ be a factor equal to the number of distinct classes represented by principle ideals.
  So it suffices to show that there are $2$ such classes if $u$ is totally positive, and $1$ otherwise.

  In the first case, if $u$ is totally positive, then $-u$ is not totally positive, and so $(u) \not \overset{+}{\sim} (-1)$, giving at least two distinct classes.
  But also there is only one non-trivial real embedding in the real quadratic case, so there is at most two distinct classes as well by $6.4.(d)$.
  So we get exactly two classes as desired.
  On the other hand, it $u$ is not totally positive, then $-u$ is totally positive, and so $(u) \overset{+}{\sim} (-1)$ by $1(u) = -u(-1)$, putting all principle ideals in a single class, which is what we needed to show.
\end{proof}


\problem{7.2}
Let $A$ be a set of primes in a number field.
Show that if $$f(s) = \prod_{P \in A} \parens*{1 - \frac{1}{||P||^s}}^{-nk}$$ has an extension in a neighborhood of $s = 1$ having a pole of order $mk$ at $s = 1$, then $$f'(s) = \prod_{P \in A} \parens*{1 - \frac{1}{||P||^s}}^{-nk}$$ has an extension with a pole of order $m$.
\begin{proof}
  Define $g(s) \coloneqq (s-1)^{mk}f(s)$, and note that this function has an analytic extension in a neighborhood of $s = 1$ since the zeros cancel the poles of $f(s)$ at $1$.
  Therefore we can define the $k$th root of $g$ analytically on the same neighborhood, to get that $g(s)^{1/k} = (s-1)^m(f(s))^{1/k} = (s-1)^m f'(s)$ is also analytic on a neighborhood of $s = 1$, which implies that $f'$ has an extension with a pole of order at most $m$ at $s = 1$.
  But also note that his pole has order at least $m$, or taking $k$th powers would give an extension of $f(s) = f'(s)^k$ with a pole of order less than $mk$.
  Therefore we conclude that $f'(s)$ has an extension with a pole of order exactly $m$ as desired.
\end{proof}


\problem{7.3}
Let $A$ and $B$ be disjoint sets of primes in a number field. Show that $$d(A \cup B) = d(A) + d(B)$$ if all of these polar densities exist, and that if any two of them exist so does the third.
\begin{proof}
  First, assume that $d(A)$ and $d(B)$ both exist, and assume $d(A) = m_A/n_A$ and $d(B) = m_B/n_B$ so that:
  \begin{equation*}
    \prod_{P \in A} \parens*{1 - \frac{1}{||P||^s}}^{-n_A} \text{ has a pole of order } m_A \text{ at } s = 1
  \end{equation*}
  \begin{equation*}
    \prod_{P \in B} \parens*{1 - \frac{1}{||P||^s}}^{-n_B} \text{ has a pole of order } m_B \text{ at } s = 1
  \end{equation*}
  Then we have that:
  \begin{equation*}
    \prod_{P \in A \cup B} \parens*{1 - \frac{1}{||P||^s}}^{-n_A n_B}
    = \parens*{\prod_{P \in A} \parens*{1 - \frac{1}{||P||^s}}^{-n_A}}^{n_B}
    \parens*{\prod_{P \in B} \parens*{1 - \frac{1}{||P||^s}}^{-n_B}}^{n_A}
  \end{equation*}
  where the right hand side has a pole of order $n_Bm_A + n_Am_B$.
  Therefore we get that $d(A \cup B)$ exists, and is equal to $(n_Bm_A + n_Am_B)/n_An_B = m_A/n_A + m_B/n_B = d(A) + d(B)$, which is what we wanted to show.

  Then for the other two cases, assume WLOG that $d(A \cup B)$ and $d(A)$ exist, and assume $d(A \cup B) = m/n$ and $d(A) = m_A/n_A$ so that:
  \begin{equation*}
    \prod_{P \in A \cup B} \parens*{1 - \frac{1}{||P||^s}}^{-n} \text{ has a pole of order } m \text{ at } s = 1
  \end{equation*}
  \begin{equation*}
    \prod_{P \in A} \parens*{1 - \frac{1}{||P||^s}}^{-n_A} \text{ has a pole of order } m_A \text{ at } s = 1
  \end{equation*}
  Then we have that:
  \begin{equation*}
    \prod_{P \in A \cup B} \parens*{1 - \frac{1}{||P||^s}}^{-n n_A}
    = \parens*{\prod_{P \in A} \parens*{1 - \frac{1}{||P||^s}}^{-n_A}}^{n}
    \parens*{\prod_{P \in B} \parens*{1 - \frac{1}{||P||^s}}^{-n n_A}}
  \end{equation*}
  Where the left hand side has a pole of order $m n_A$.
  Therefore $d(B)$ must exist since the right hand side must also have an extension in the neighborhood, and by counting poles we get that $d(B)$ is equal to $(mn_A - m_An)/nn_A = m/n - m_A/n_A = d(A \cup B) - d(A)$ as desired.
\end{proof}


\problem{7.15}
\subproblem{a}
Let $G$ be a cyclic group of order $n$. Show that the character group $\hat G$ is also cyclic of order $n$
\begin{proof}
  We can assume that $G = \Z/n\Z$ since cyclic groups are all isomorphic, and note that each character on $G$ is determined by $\chi(1)$ since $1$ generates $\Z/n\Z$.
  Furthermore, the order of $\chi(1)$ must divide the order of $1$, so it must be a $n$th root of unity, and conversely any $n$th root of unity defines a character sending $1$ to that root of unity.
  Hence $\hat{G}$ is in bijection with the $n$th roots of unity, and $|\hat{G}| = n$, so it just remains to show that $\hat{G}$ is cyclic.

  Fix some primitive $n$th root of unity $\zeta_0$, and let $\chi_0$ be defined by $\chi_0(1) = \zeta_0$.
  For any other character $\chi$, let $k$ be the power such that $\zeta_0^k = \chi(1)$.
  Then we have $\chi_0^k(1) = (\chi_0(1))^k = \zeta_0^k = \chi(1)$, so that $\chi_0^k = \chi$, and every other character is a power of $\chi_0$, and we conclude that $\hat{G}$ is cyclic.
\end{proof}

\subproblem{b}
Let $G$ and $H$ be finite abelian groups. Show that there is an isomorphism $\hat{G} \times \hat{H} \rightarrow \widehat{G \times H}$.
\begin{proof}
  Define $\varphi : \hat{G} \times \hat{H} \rightarrow \widehat{G \times H}$ by $\varphi(\chi_G, \chi_H)(x, y) = \chi_G(x)\chi_H(y)$ for all $(x,y) \in G \times H$.
  We argue that $\varphi$ gives the desired isomorphism.

  For surjectivity, assume $\chi \in \widehat{G \times H}$, and define $\chi_G \in \hat G$ by $\chi_G(x) = \chi(x, 1)$ and $\chi_H \in \hat H$ by $\chi_H(y) = \chi(1, y)$.
  Then for any $(x,y) \in G \times H$, we have $\varphi(\chi_G, \chi_H)(x, y) = \chi_G(x)\chi_H(y) = \chi(x,1)\chi(1,y) = \chi(x,y)$, which by extensionality implies that $\varphi(\chi_G, \chi_H) = \chi$, making $\varphi$ surjective.

  For injectivity, it suffices to show that the kernel is trivial, so assume that $\varphi(\chi_G, \chi_H) = 1_{\widehat{G \times H}}$.
  Therefore $\chi_G(x)\chi_H(y) = 1$ for all $(x,y) \in G \times H$.
  In particular, if we set $x = 1$ then $\chi_H(y) = \chi_G(1)\chi_H(y) = 1$ for all $y \in H$ and if we set $y = 1$ $\chi_G(x) = \chi_G(x)\chi_H(1) = 1$ for all $x \in G$.
  Therefore we conclude $\chi_G = 1_{\hat G}$ and $\chi_H = 1_{\hat H}$, and the kernel is indeed trivial, and we conclude $\varphi$ is also injective, completing the proof.
\end{proof}

\subproblem{c}
Let $G$ be a finite abelian group. Prove that $\hat G$ is isomorphic to $G$.
\begin{proof}
  By the fundamental theorem of finite abelian groups, we can write $G \cong \prod_i \Z/p_i\Z$ for some finite set of primes $p_i$ not necessarily distinct.
  Then using the results of parts (a) and (b) (note that (b) applies inductively to finite products) we have:
  \begin{equation*}
    \hat{G}
    \cong \widehat{\prod_i \Z/p_i\Z}
    \cong \prod_i \widehat{\Z/p_i\Z}
    \cong \prod_i \Z/p_i\Z \cong G
  \end{equation*}
\end{proof}


\problem{7.16}
Prove that for nontrivial even character mod $m$ that $\sum_{k \in (\Z/m\Z)^\times} \chi(k) k = 0$, and for an odd character $\sum_{k \in (\Z/m\Z)^\times} \chi(k) \log \sin \frac{k\pi}{m} = 0$.
\begin{proof}
  First, in the even character case, we have $\chi(m-k) = \chi(m) + \chi(-k) = 0 + \chi(k) = \chi(k)$.
  Also note that $k \mapsto m - k$ is an automorphism of $(\Z/m\Z)^\times$, so:
  \begin{equation*}
    \sum_{k \in (\Z/m\Z)^\times} \chi(k) k = \sum_{k \in (\Z/m\Z)^\times} \chi(m - k)(m-k) = \sum_{k \in (\Z/m\Z)^\times} \chi(k)(m-k)
  \end{equation*}
  And using this we have:
  \begin{equation*}
    2\sum_{k \in (\Z/m\Z)^\times} \chi(k) k
    = \sum_{k \in (\Z/m\Z)^\times} \chi(k) k + \sum_{k \in (\Z/m\Z)^\times} \chi(k) (m-k)
    = \sum_{k \in (\Z/m\Z)^\times} \chi(k) (k + m - k)
    = m\sum_{k \in (\Z/m\Z)^\times} \chi(k)
    = 0
  \end{equation*}
  and we conclude that $\sum_k \chi(k) k = 0$ as desired.

  For the odd case, we have $\chi(m - k) = \chi(m) + \chi(-k) = -\chi(k)$, and so again using the fact that $k \mapsto m - k$ is an automorphism, we calculate:
  \begin{align*}
    \sum_{k \in (\Z/m\Z)^\times} \chi(k) \log \sin \frac{k\pi}{m} 
    &= \sum_{k \in (\Z/m\Z)^\times} \chi(m-k) \log \sin \frac{(m-k)\pi}{m} \\
    &= \sum_{k \in (\Z/m\Z)^\times} -\chi(k) \log \sin \frac{k\pi}{m} \\
    &= - \sum_{k \in (\Z/m\Z)^\times} \chi(k) \log \sin \frac{k\pi}{m} 
  \end{align*}
  and so we conclude that $\sum_{k \in (\Z/m\Z)^\times} \chi(k) \log \sin \frac{k\pi}{m} = 0$, since $y = -y$ iff $y = 0$.
\end{proof}


\problem{7.17}
Let $m$ be even, $m \ge 3$, and suppose $\chi$ is a primitive character mod $m$. Set $n = m/2$.

\subproblem{a}
Show that $n$ must be even.
\begin{proof}
  It suffices to show that there are no primitive characters mod $2n$ for $n$ odd, so let $\chi$ be a character mod $2n$ for some odd $n$.
  Then since $n$ is odd, $2$ and $n$ are relatively prime, and so $\varphi(2n) = \varphi(2)\varphi(n) = \varphi(n)$ where $\varphi$ is the Euler totient function.
  But then we have $(\Z/2m\Z)^\times \cong (\Z/m\Z)^\times$ since $m | 2m$ and the groups have the same size (by above).
  But then $\chi$ factors through this isomorphism and so is induced by a character mod $n$, and so $\chi$ is not primitive.
  Therefore since there are no primitive characters mod $m = 2n$ for $n$ odd, we conclude that $n$ must be even.
\end{proof}

\subproblem{b}
Show that $\equivmod{(n+1)^2}{1}{m}$, and for odd $k$, $\equivmod{(n+1)k}{n+k}{m}$.
\begin{proof}
  Since $n$ is even, we can write $n = 2n'$ for some $n'$, and so:
  \begin{equation*}
    (n+1)^2 = n^2 + 2n + 1 = \equivmod{mn' + m + 1}{1}{m}
  \end{equation*}
  If we further assume $k$ is odd, so $k = 2k'+1$ for some $k'$, then:
  \begin{equation*}
    (n+1)k = n(2k'+1) + k = \equivmod{mk' + n + k}{n + k}{m}
  \end{equation*}
\end{proof}

\subproblem{c}
Prove that $\chi(n + 1) = -1$.
\begin{proof}
  using (b) and the fact that $\chi$ is defined in elements mod $m$, we have $\chi(n + 1)^2 = \chi((n+1)^2) = \chi(1) = 1$, so $\chi(n+1) = \pm 1$.
  But if $\chi(n+1) = 1$, then $chi$ is just the trivial character, which isn't primitive, so we must have $\chi(n+1) = -1$ as desired.
\end{proof}

\subproblem{d}
Prove that $\chi(n + k) = - \chi(k)$ for odd $k$.
\begin{proof}
  Using (b) and (c) we have:
  \begin{equation*}
    \chi(n+k) = \chi((n+1)k) = \chi(n+1)\chi(k) = -\chi(k)
  \end{equation*}
  as desired
\end{proof}


\problem{7.39}

Let $\chi_1, \dots, \chi_n$ be the even characters mod $p$, $\chi_1 = 1$, and let $X$ be the $n \times n$ matrix $X = \parens*{\chi_i(k)}$ with $\chi_i(k)$ in row $i$ and column $k$.
Prove that $det(X) \ne 0$.

\begin{proof}
  Let $\alpha$ be a generator for $(\Z/p\Z)^\times$, and $e_n$ the power such that $\alpha^{e_n} = n$.
  Then we have that:
  \begin{equation*}
    \det(X) = \det \parens*{\chi_i(k)}
    = \det \parens*{\chi_i(\alpha^{e_k})}
    = \det \parens*{\chi_i(\alpha)^{e_k}}
  \end{equation*}
  Which is a vandermonde matrix, up to swapping the column orderings to order the $e_k$ values.
  Therefore the determinant is a vandermonde determinant up to swapping signs, and we have $\det(X) = \pm \prod_{1 \le r < s \le n} (\chi_s(\alpha) - \chi_r(\alpha))$.
  But then all the terms of the product are nonzero, since for $s \ne r$ we have $\chi_s(\alpha) \ne \chi_r(\alpha)$, since the characters are determined by there value on the generator $\alpha$.
  So finally we conclude that $\det(X) \ne 0$ since a product of non-zero terms is non-zero in $\C$.
\end{proof}


\problem{8.1}
Let $L$ and $L'$ e two finite normal extensions of a number field $K$ and let $A$ and $A'$ be the sets of primes of $K$ which split completely in $L$ and $L'$. suppose $A$ and $A'$ differ by a set of polar density $0$. Show that $L = L'$.
\begin{proof}
  Since $A$ splits completely in $L$ and $A'$ splits completely in $L'$, we have that $A \cap A'$ splits completely in $L$ and $L'$, and so splits completely in $LL'$ by theorem 31.
  But then note that since $A$ and $A'$ differ by a set of polar density $0$, the same is true of $A \cap A'$ and $A$ and of $A \cap A'$ and $A'$.
  Therefore by theorem 43 $L$, $L'$ and $LL'$ all have the same degree over $K$.
  Finally then, the extensions $LL'/L$ and $LL'/L'$ must both be trivial, and we get $L = LL' = L'$ as desired.
\end{proof}


\problem{8.3}
Let $\Pi$ be a free abelian semi-group and let $f : \Pi \rightarrow G$ be a homomorphism onto a finite abelian group $G$.
Show that the kernel of $f$ is a closed semi-group in $\Pi$ and that there is a one-to-one correspondence between subgroups of $G$ and the closed semi-groups in $\Pi$ which contain the kernel.
\begin{proof}
  First we show that $\ker f$ is closed, so assume $I, J \in \ker f$, or equivalently that $f(I) = f(J) = 0$.
  Then since $f$ is a homomorphism, we have $f(IJ) = f(I) + f(J) = 0 + 0 = 0$, and so $IJ \in \ker f$ as well, and the kernel is closed.
  Also $1 \in \ker f$, since $f$ being a homomorphism implies $f(1) = 0$.
  Therefore we conclude $\ker f$ is a closed semi-group of $\Pi$.

  Then consider the map sending a closed semi-group $\mathbb{S}$ in $\Pi$ to the image $f(\mathbb{S})$, which is a subset of $G$.
  First we show that these images $f(\mathbb{S})$ are subgroups of $G$ for $\mathbb{S}$ containing $\ker f$.
  $f(\mathbb{S})$ contains $0 = f(1)$, since $\ker f \le \mathbb{S}$ contains $1$.
  $f(\mathbb{S})$ is also closed under addition, since if $x, y \in f(\mathbb{S})$, so that $x = f(I)$ and $y = f(J)$, then $IJ \in \mathbb{S}$ since $\mathbb{S}$ is closed, and so $f(IJ) = f(I) + f(J) = x + y$ is in $f(\mathbb{S})$.
  Finally, it is closed under inverses, since in a finite abelian group the additive inverse of $x$ is some multiple of $x$ ($x$ has finite order since the group is finite, and so $1$ is some multiple of it), and so closure under addition gives closure under inverses.
  
  We argue that the above mapping gives the desired correspondence, and by the above argument it just remains to show that this mapping is one-to-one.
  So assume $\mathbb{S}_1$ and $\mathbb{S}_2$ are closed semi-groups containing $\ker f$, and that $f(\mathbb{S}_1) = f(\mathbb{S}_2)$.
  Then for any $I \in \mathbb{S}_1$, there is some $J \in \mathbb{S}_2$ such that $f(I) = f(J)$.
  But then $I$ and $J$ differ only by an element of $\ker f$, and so since both semi-groups contain $\ker f$, they differ by an element of the semi-groups.
  Hence by the closure of $\mathbb{S}_2$, we must have $I$ in $\mathbb{S}_2$ as well, and so $\mathbb{S}_1 \le \mathbb{S}_2$.
  A similar argument gives the inclusion in the reverse direction, and we conclude $\mathbb{S}_1 = \mathbb{S}_2$, making the map one-to-one as desired.  
\end{proof}


\problem{8.20}
Determine the Hilbert class field over $\Q[\sqrt{m}]$ for $m = -6, -10, -21, -30$.
\begin{proof}
  For $\Q[\sqrt{-6}]$, we know from exercise 5.11 that the class number is $2$, so the Hilbert class field is an extension of degree $2$, and hence of the form $\Q[\sqrt{-6}, \sqrt{d}]$ for some $d \in \Q[\sqrt{-6}]$.
  Then since $\equivmod{-6/2}{1}{4}$, we have by exercise 8.15 that $d = 2$ makes this extension unramified, so since the Hilbert class field is the maximal unramified extension it must contain $\Q[\sqrt{-6}, \sqrt{2}]$.
  But this is already a degree $2$ extension, and the class number is $2$, so this is actually the Hilbert class field: $H = \Q[\sqrt{-6}, \sqrt{2}] = \Q[\sqrt{-3}, \sqrt{2}]$.

  For $\Q[\sqrt{-10}]$ we know again from exercise 5.11 that the class number is $2$, and so the extension is of the form $\Q[\sqrt{-10}, \sqrt{d}]$.
  Again by exercise 8.15 we find that $d = 5$ makes this unramified, so for the same reasons as before we conclude that $H = \Q[\sqrt{-10}, \sqrt{5}] = \Q[\sqrt{-2}, \sqrt{5}]$.

  For $\Q[\sqrt{-21}]$ and $\Q[\sqrt{-30}]$, the ideal class group is isomorphic to the Klein-4 group by exercise 5.14, so the Hilbert class field is a degree $4$ extension, and its Galois group must be isomorphic to Klein-4.
  By checking all the possible totally unramified extensions, we see that these are $\Q[\sqrt{-3},\sqrt{-7}]$ and $\Q[\sqrt{-2},\sqrt{15}]$.
\end{proof}


\problem{8.19}
Show that $h^+ = 6$ for $\Q[\sqrt{223}]$.
\begin{proof}
  We know that $h = 3$ for $\Q[\sqrt{223}]$ by exercise $5.8$, and so by exercise $6.9$ we know that $h^+ = 6$ if the fundamental unit is totally positive, and $h^+ = 3$ otherwise, and so it suffices to check that the fundamental unit is totally positive.

  Using the code from problem $5.34$, we get that the fundamental unit is $u = 224 + 15\sqrt{223}$.
  Then the only non-trivial automorphism $\sigma \in Gal(\Q[\sqrt{223}]/\Q)$ is the one given by $\sqrt{223} \mapsto -\sqrt{223}$.
  But we have $\sigma(u) = \sigma(224 + 15\sqrt{223}) = 224 - 15\sqrt{223} = 224 - 223.997... > 0$.
  Hence $u$ is totally positive, and we conclude that $h^+ = 6$ as desired.
\end{proof}


\problem{8.25}
Let $K \subset L$ be number fields, $L$ normal over $K$, and let $H^+$ denote the Hilbert+ class field over $L$. Prove that $H^+$ is a normal extension of $K$.
\begin{proof}
  Consider the normal closure $N$ of $H^+$ over $K$.
  The normal closure is constructed by adjoining conjugates of elements of the base field, and so $N$ is a unramified extension of $L$.
  But then we know that $H^+$ is the maximal unramified extension of $L$, so $H^+$ must contain its normal closure $N$.
  Therefore we conclude that $H^+$ is already a normal extension, as desired.
\end{proof}

\end{document}
