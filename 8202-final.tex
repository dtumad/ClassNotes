%&pdflatex
\documentclass[11pt]{article}

\usepackage{main-macros}


\usepackage[margin=0.75in]{geometry}


\title{8202 Final}
\author{Devon Tuma}
\date{Spring 2020}

\begin{document}
\maketitle

\problem{1}
\subproblem{a: false}
\begin{proof}
  $V_4$ has two subgroups of order $2$, which must correspond to quadratic extensions.
  We know that quadratic extensions are extensions by the squareroot of a squarefree element, so we must have $\K = \Q(\sqrt{D_1}, \sqrt{D_2})$.
  TODO: Therefore $f(x)$ must split into quadratic factors?
\end{proof}

\subproblem{b: true}
\begin{proof}
  Since the extension is Galois, it must be the splitting field for some polynomial $g(x)$.
  But by the previous problem, $g(x)$ can't be irreducible, so it must be reducible.
\end{proof}

\subproblem{c}
\subproblem{d}
\subproblem{e}
\subproblem{f}
\subproblem{g}
\subproblem{h}


\problem{2}
\subproblem{a: 3}
\begin{proof}
  We know that the $\R[x]$-submodules of $V$ are in bijection with the subpaces of $V$ stable under the linear transformation for this matrix.
  All non-trivial sub-spaces will have dimension one, so we only need to count the stable one-dimensional subspaces.
  But note that elements of a one-dimensional subspace are all scalar multiples of each other,
  and this matrix sends $\begin{bmatrix}a & b\end{bmatrix}$ to $\begin{bmatrix}b & a\end{bmatrix}$,
  so the only stable subspace is the one generated by $\begin{bmatrix}1 & 1\end{bmatrix}$.
  This gives exactly one corresponding submodule, and so including the two trivial submodules we conlude there are three in total.
\end{proof}

\subproblem{b: $2^n$}
\begin{proof}
  Again this is the same as counting the number of $A$ stable subspaces of $V$.
  First note that the roots of the characteristic polynomial $\det(xI - A) = x^n - 1$ are exactly the $n$th roots of unity.
  Then these roots are also the eigenvalues of $A$, so $A$ has $n$ distinct eigenvalues and hence is diagonalizable.
  In particular $V$ has a basis consisting of $n$ distinct linearly independent eigenvectors of $A$, call them $v_1,\dots,v_n$.
  But note that any set of eigenvectors generates a stable subspace, since $A$ just acts by scaling the generating eigenvectors.
  So the number of stable subspaces is at least the number of subsets of these $n$ eigenvectors.
  
  We then claim there can't be any more stable subspaces than this.
  Given a stable subspace $W$, we know that we can find a basis of eigenvectors for it, say $w_1,\dots,w_k$.
  Then $w_i$ must be a linear combination of the $v_1,\dots,v_n$, since they were a basis for all of $V$.
  But each of those eigenvectors had distinct eigenvalues, so $w_i$ can't be an eigenvector if it is a combination of more than one of them.
  Hence $w_i = kv_j$ for one of the $v_j$, and this holds for all of the $w_i$, and hence $W$ is generated by a subset of the $v_1,\dots,v_n$.
  Therefore the number of stable subspaces is at most the number of subsets of the previous $n$ eigenvectors.
  
  Finally then the number of $\C[x]$-submodules is the number of subsets of $n$ elements, which is $2^n$.
\end{proof}

\problem{3}
\begin{proof}
  page 24 of notes batch 5
\end{proof}

\problem{4}
\subproblem{a}
\subproblem{b}
\subproblem{c}


\end{document}