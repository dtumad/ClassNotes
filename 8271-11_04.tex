%&pdflatex
\documentclass[11pt]{article}

\usepackage{main-macros}

\usepackage{newpxmath}
\usepackage{newpxtext}
\usepackage[margin=0.75in]{geometry}


\title{8271 11/04 Reading}
\author{Devon Tuma}
\date{Fall 2020}

\begin{document}
\maketitle

\section*{Question Answering}

\begin{itemize}
\item [1] Why is exploit generation particularly challenging in PHP?

  Many elements of a page are generated dynamically, and are therefore hard to understand without understanding how the code will execute.
  Even code itself can often be generated on the fly by the website, which makes analysis in general very difficult.
  There is also the fact that different parts of code and generated code are executed at different tiers that can matter to how they affect each other.
  
\item [2] How would the built-in functions affect the symbolic execution of PHP, and how are built-in functions handled?

  The symbolic execution only targets the modules that contain pontentially vulnerable sinks, and does not prioritize the rest.
  Certain built-in functions are also tagged as sanitizing for certain sanitization patterns, based on their semantics.
  NAVEX also needs solver specifications for the language, but in general NAVEX is applicable to any system of built-in functions with such a specification.
\end{itemize}

\section*{Paper Critiques}

\subsection*{Short Summary}

The paper disucusses a new method to generate exploits in code that contains many dynamic aspects, that uses methods from static analysis and extends them.
This combination of dynamic and static component makes both run more efficiently to better generate vulnerabilites.

\subsection*{Limitations of the paper}

The system does not have a full implementation of file upload capabilities, which is a very important potential attack vector.
Adding this in would therefore be a very valuable change

\subsection*{Potential follow-up work}

There are still a number of false positives that still occur, so one follow-up work that might be worthwhile is looking more closely at trying to eliminate these and understand where they are coming from.

\end{document}
