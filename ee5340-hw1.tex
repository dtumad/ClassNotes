%&pdflatex
\documentclass[11pt]{article}

\usepackage{main-macros}
\usepackage{listings}

\usepackage{newpxmath}
\usepackage{newpxtext}
\usepackage[margin=0.75in]{geometry}


\title{EE5340 HW1}
\author{Devon Tuma}
\date{Sprint 2021}

\begin{document}
\maketitle

\section*{Source Code}

\begin{lstlisting}[language=R]
  library("QuantumOps")

  printKet <- function(k) {
    print(k)
    print(dirac(k))
    print(probs(k))
  }
  
  k1 <- ket(1, 0)
  print("Problem 1")
  printKet(k1)
  print("Problem 2")
  printKet(X(k1))
  
  k2 <- ket(sqrt(1/3), sqrt(2/3))
  print("Problem 3.1")
  printKet(k2)
  print("Problem 3.2")
  printKet(X(k2))
\end{lstlisting}

\section*{Code Output}

The results of the X gate applications both make sense because they swap the probabilities of observing each of the corresponding states, as seen below in the code output:

\begin{verbatim}
[1] "Problem 1"
     [,1]
[1,] 1+0i
[2,] 0+0i
[1] "1|0>"
     [,1]
[1,]    1
[2,]    0
[1] "Problem 2"
     [,1]
[1,] 0+0i
[2,] 1+0i
[1] "1|1>"
     [,1]
[1,]    0
[2,]    1
[1] "Problem 3.1"
             [,1]
[1,] 0.5773503+0i
[2,] 0.8164966+0i
[1] "0.577|0> + 0.816|1>"
          [,1]
[1,] 0.3333333
[2,] 0.6666667
[1] "Problem 3.2"
             [,1]
[1,] 0.8164966+0i
[2,] 0.5773503+0i
[1] "0.816|0> + 0.577|1>"
          [,1]
[1,] 0.6666667
[2,] 0.3333333
\end{verbatim}

\end{document}
