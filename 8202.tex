%&pdflatex
\documentclass[11pt]{article}

\usepackage{main-macros}

\usepackage{newpxmath}
\usepackage{newpxtext}
\usepackage[margin=0.75in]{geometry}


\title{8202 Notes}
\author{Devon Tuma}
\date{Spring 2020}

\begin{document}
\maketitle

\section*{Splitting Fields and Algebraic Closures}

\begin{definition}
  $K/F$ is a splitting field for $f(x) \in F[x]$ if you can factor $f(x)$ into a product of linear polynomials, i.e.\ if all roots are in $K$, and $K$ is the smallest such field.
  This generalizes to the splitting field for a set $\{f_i(x)\}_{i \in I}$ in the obvious way.
  In general splitting field is $K = F(\alpha_i)$ for all roots $\alpha_i$ of the relevant polynomial(s).
  If $K$ is a splitting field for some polynomial we say it is a normal extension.
\end{definition}

\begin{theorem}
  $F$ a field, $f(x) \in F[x]$  a polynomial of degree $n$, then there exists a splitting field $K$ for $f(x)$ over $F$ of degree $\le n!$.
\end{theorem}
\begin{proof}
  By induction, we can assume this is possible for $f$ of degree less than $n$.
  Then choose some root $\alpha$ of $f$ not in $F$, and adjoin it on.
  Then $f$ factors as $(x - \alpha)\hat{f}(x)$ in $F(\alpha)$, with $\deg(\hat{f}) < \deg(f)$.
  Thus by induction we can complete the factorization to get a splitting field
\end{proof}

\begin{definition}
  $K/F$ is an algebraic closure of $F$ (written $K := \bar{F}$) if $K/F$ is algebraic and every $f(x) \in F[x]$ splits completely in $K$.
  If $F = \bar{F}$ we say that $F$ is algebraically closed.
\end{definition}

\begin{theorem}
  The algebraic closure of a field is algebraically closed, and every field lies inside some algebraic closure.
\end{theorem}
\begin{proof}
  The idea is to first show $F$ is contained in a $K_1$ that has a root for every $f(x) \in F[x]$. We take $F[\{y_f : f \in F[x]\}]$ and quotient by a maximal ideal generated by all monic polynomials in $F[x]$, so the image $\alpha_f$ of $y_f$ becomes a root for $f$ in this field. You can then continue this process by creating $K_{i+1}$ with a root for all $f \in K_i[x]$. Then taking the union of all $K_i$ we get the desired $K$.
\end{proof}

\begin{example}
  $\complexes$ isn't the algebraic closure of $\rationals$, it is too big. In particular, the closure of $\rationals$ is a countable set because the set of polynomials in rational coefficients is countable. $\complexes$ is however the closure of $\reals$.
\end{example}

\begin{example}
  The algebraic closure of $\finfield{2}$ is $\cup_{d \in \integers^+} \finfield{2^d}$ because of the way finite fields are contained within each other.
\end{example}

\begin{example}
  The algebraic closure of $\complexes(t)$ will contain things like $\sqrt[5]{t}$ and $\sqrt[3]{t + \sqrt{t}}$. Also contains roots of quintic and higher functions so that they can't be expressed only via radicals.
\end{example}

\begin{theorem}[Isomorphism Extension Theorem]
  Given an isomorphism $\varphi : F \rightarrow F'$ with $\{f_i\}_{i \in I} \subseteq F[x]$ so $\varphi(f_i) \in F'[x]$, then if $K$ is a splitting field over $F$ for $\{f_i\}_{i \in I}$ and $K'$ is a splitting field for $\{\varphi(f_i)\}_{i \in I}$ over $F'$ then $\varphi$ extends to an isomorphism from $K$ to $K'$. Furthermore this extension can permute the roots of minimal polynomials in any way and still be an isomorphism.
\end{theorem}
\begin{remark}
  This theorem can be used to generate automorphisms of splitting fields by permuting the roots of minimal polynomials. This will be important when calculating Galois groups.
\end{remark}


\end{document}
