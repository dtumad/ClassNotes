%&pdflatex
\documentclass[11pt]{article}

\usepackage{main-macros}

\usepackage{newpxmath}
\usepackage{newpxtext}
\usepackage[margin=0.75in]{geometry}


\title{8202 Notes}
\author{Devon Tuma}
\date{Spring 2020}

\begin{document}
\maketitle

\section*{Splitting Fields and Algebraic Closures}

\begin{definition}
  $K/F$ is a splitting field for $f(x) \in F[x]$ if you can factor $f(x)$ into a product of linear polynomials, i.e. if all roots are in $K$, and $K$ is the smallest such field.
  This generalizes to the splitting field for a set $\{f_i(x)\}_{i \in I}$ in the obvious way.
  In general splitting field is $K = F(\alpha_i)$ for all roots $\alpha_i$ of the relevant polynomial(s).
  If $K$ is a splitting field for some polynomial we say it is a normal extension.
\end{definition}

\begin{theorem}
  $F$ a field, $f(x) \in F[x]$  a polynomial of degree $n$, then there exists a splitting field $K$ for $f(x)$ over $F$ of degree $\le n!$.
\end{theorem}

\begin{proof}
  By induction, we can assume this is possible for $f$ of degree less than $n$.
  Then choose some root $\alpha$ of $f$ not in $F$, and adjoin it on.
  Then $f$ factors as $(x - \alpha)\hat{f}(x)$ in $F(\alpha)$, with $\deg(\hat{f}) < \deg(f)$.
  Thus by induction we can complete the factorization to get a splitting field
\end{proof}

\begin{definition}
  $K/F$ is an algebraic closure of $F$ (written $K := \bar{F}$) if $K/F$ is algebraic and every $f(x) \in F[x]$ splits completely in $K$.
  If $F = \bar{F}$ we say that $F$ is algebraically closed.
\end{definition}

\begin{theorem}
  The algebraic closure of a field is algebraically closed, and every field lies inside some algebraic closure.
\end{theorem}

\begin{remark}
  $\complexes$ isn't the algebraic closure of $\rationals$, it is too big. It is however the closure of $\reals$.
\end{remark}


\end{document}
