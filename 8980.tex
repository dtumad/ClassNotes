%&pdflatex
\documentclass[11pt]{article}

\usepackage{main-macros}

\usepackage{newpxmath}
\usepackage{newpxtext}
\usepackage[margin=0.75in]{geometry}


\title{8980 Notes}
\author{Devon Tuma}
\date{Spring 2020}

\begin{document}
\maketitle

\section*{Inductive Data Types}

\begin{definition}[Inductive Data Type]
  Data type freely generated by a number of generators. Note that generators like $suc : \naturals \rightarrow \naturals$ are okay but $fsuc : (\naturals \rightarrow \naturals) \rightarrow \naturals$ are not, as they lead to paradoxes.
  In general the domain of a constructor must be covariant in the type being defined.
\end{definition}

\begin{example}
  $\naturals$ with constructors $zero : \naturals$ and $suc : \naturals \rightarrow \naturals$. 
\end{example}

\begin{example}
  $\mathbb{S}^1$ with constructors $base : \mathbb{S}^1$ and $loop : Id(base ; base)$.
\end{example}

This definition of the circle is more complicated than other types, because we've introduced a path, not just new elements.
Therefore this is a \textbf{higher inductive type}.
Writing down the rules for this type is more work than for regular inductive types.
In order to make sense of the uniqueness rules, we'll first have to define a dependent version of $ap$.
While $ap$ allows us to apply a function to a path to get a path in the codomain, what we need now is a way to apply a dependent function to a path to get a path in the total space, where we think of a path in the total space as a path over the path in the base space.


After doing that we are able to give the rules of the sphere type:

\begin{multicols}{3}
  \noindent
  \begin{equation*}
    \begin{prooftree}
      \infer0[] {
        \mathbb{S}^1 : U
      }
    \end{prooftree}
  \end{equation*}
  \begin{equation*}
    \begin{prooftree}
      \infer0[] {
        base : \mathbb{S}^1
      }
    \end{prooftree}
  \end{equation*}
  \begin{equation*}
    \begin{prooftree}
      \infer0[] {
        loop : Id_{\mathbb{S}^1}(base; base)
      }
    \end{prooftree}
  \end{equation*}
\end{multicols}


\begin{remark}
  We can formalize the idea of a type being freely generated by considering initial objects in appropriate algebra categories.
  For example for $\naturals$ we can think of the category where objects are types equipped with a zero point and successor function, and arrows are function (morphisms) that respect the zero and successor functions in the domain and codomain (think group homomorphism respecting group operations).
  In this category, the freely generated inductive object is the initial object of the category.
\end{remark}


\end{document}
