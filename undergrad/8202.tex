%&pdflatex
\documentclass[11pt]{article}

\usepackage{main-macros}

\usepackage{newpxmath}
\usepackage{newpxtext}
\usepackage[margin=0.75in]{geometry}


\title{8202 Notes}
\author{Devon Tuma}
\date{Spring 2020}

\begin{document}
\maketitle

\section*{Splitting Fields and Algebraic Closures}

\begin{definition}
  $\K/\F$ is a splitting field for $f(x) \in \F[x]$ if you can factor $f(x)$ into a product of linear polynomials, i.e.\ if all roots are in $\K$, and $\K$ is the smallest such field.
  This generalizes to the splitting field for a set $\{f_i(x)\}_{i \in I}$ in the obvious way.
  In general splitting field is $\K = \F(\alpha_i)$ for all roots $\alpha_i$ of the relevant polynomial(s).
  If $\K$ is a splitting field for some polynomial we say it is a normal extension.
\end{definition}

\begin{theorem}
  $\F$ a field, $f(x) \in \F[x]$  a polynomial of degree $n$, then there exists a splitting field $\K$ for $f(x)$ over $\F$ of degree $\le n!$.
\end{theorem}
\begin{proof}
  By induction, we can assume this is possible for $f$ of degree less than $n$.
  Then choose some root $\alpha$ of $f$ not in $\F$, and adjoin it on.
  Then $f$ factors as $(x - \alpha)\hat{f}(x)$ in $\F(\alpha)$, with $\deg(\hat{f}) < \deg(f)$.
  Thus by induction we can complete the factorization to get a splitting field
\end{proof}

\begin{definition}
  $\K/\F$ is an algebraic closure of $\F$ (written $\K := \bar{\F}$) if $\K/\F$ is algebraic and every $f(x) \in \F[x]$ splits completely in $\K$.
  If $\F = \bar{\F}$ we say that $\F$ is algebraically closed.
\end{definition}

\begin{theorem}
  The algebraic closure of a field is algebraically closed, and every field lies inside some algebraic closure.
\end{theorem}
\begin{proof}
  The idea is to first show $\F$ is contained in a $\K_1$ that has a root for every $f(x) \in \F[x]$. We take $\F[\{y_f : f \in \F[x]\}]$ and quotient by a maximal ideal generated by all monic polynomials in $\F[x]$, so the image $\alpha_f$ of $y_f$ becomes a root for $f$ in this field. You can then continue this process by creating $\K_{i+1}$ with a root for all $f \in \K_i[x]$. Then taking the union of all $\K_i$ we get the desired $\K$.
\end{proof}

\begin{example}
  $\C$ isn't the algebraic closure of $\Q$, it is too big. In particular, the closure of $\Q$ is a countable set because the set of polynomials in rational coefficients is countable. $\C$ is however the closure of $\R$.
\end{example}

\begin{example}
  The algebraic closure of $\F_{2}$ is $\cup_{d \in \Z^+} \F_{2^d}$ because of the way finite fields are contained within each other.
\end{example}

\begin{example}
  The algebraic closure of $\C(t)$ will contain things like $\sqrt[5]{t}$ and $\sqrt[3]{t + \sqrt{t}}$. Also contains roots of quintic and higher functions so that they can't be expressed only via radicals.
\end{example}

\begin{theorem}[Isomorphism Extension Theorem]
  Given an isomorphism $\varphi : \F \rightarrow \F'$ with $\{f_i\}_{i \in I} \subseteq \F[x]$ so $\varphi(f_i) \in \F'[x]$, then if $\K$ is a splitting field over $\F$ for $\{f_i\}_{i \in I}$ and $\K'$ is a splitting field for $\{\varphi(f_i)\}_{i \in I}$ over $\F'$ then $\varphi$ extends to an isomorphism from $\K$ to $\K'$. Furthermore this extension can permute the roots of minimal polynomials in any way and still be an isomorphism.
\end{theorem}
\begin{proof}
  Proven by induction on the extension degree $[\K : \F]$. Base case of degree $1$ extension is simple, since then $\K = \F$, which implies $\K' = \F'$. For the inductive case, adjoin the $\alpha$ to $\F$ where $\alpha$ is the root of some polynomial that doesn't split in $\F$, and adjoint to $\F'$ a root $\alpha'$ of $m_\alpha' = \varphi(m_\alpha)$. Then $\F(\alpha) \cong F(\alpha')$, since $\F[x]/(m_\alpha(x)) \cong F'[x]/(m_\alpha')$. Then this gives us an extension with one less degree, so we can apply the inductive hypothesis to this new isomorphism to get the desired isomorphism from $\K$ to $\K'$.
\end{proof}
\begin{remark}
  Morandi's book theorem 3.20 gives a proof for the case where the extension degree is $\infty$ using Zorn's lemma on pairs of towers.
\end{remark}
\begin{remark}
  This theorem can be used to generate automorphisms of splitting fields by permuting the roots of minimal polynomials. This will be important when calculating Galois groups.
\end{remark}

\begin{corollary}
  Splitting fields $\K/\F$ for any set of polynomials are unique up to isomorphism. Therefore we can talk of \textit{the} splitting field $\K = \text{Split}_F(\{f_i\})$.
\end{corollary}
\begin{proof}
  Apply the isomorphism extension theorem to the identity map on the base field $\F$.
\end{proof}

\begin{definition}
  $Aut(K/F)$ is the set of field automorphisms from $\K$ to $\K$ fixing $\F$. Also called $Gal(K/F)$ if this is a Galois extension.
\end{definition}
\begin{example}
  Let $\K = \text{Split}_\Q(x^4 - 7)$. Then $\varphi \in Aut(K/Q)$ is determined by $\varphi(i) \in \{\pm i\}$ and $\varphi(\sqrt[4]{7} \in \{\pm\sqrt[4]{7}, \pm i\sqrt[4]{7}\}$, so the automorphism group has size $8$. In particular $Aut(K/Q) \cong D_8$. The map sending $\sqrt[4]{7}$ to $i\sqrt[4]{7}$ acts as a rotation by $90$ degrees, where we imagine the corners of the square being $\{\pm\sqrt[4]{7}, \pm i\sqrt[4]{7}\}$. Similarly the map sending $i$ to $-i$ while fixing $\sqrt[4]{7}$ is the reflection map.
\end{example}
  
\begin{definition}
  An extension $\K/\F$ is a normal extension if $\K = split_F(\{f_i\}_{i \in I})$ for some set of polynomials $f_i$.
\end{definition}
\begin{theorem}
  The following equivalent conditions are equivalent:
  \begin{itemize}
  \item[(i)] $\K/\F$ is normal
  \item[(ii)] Every nonzero field homomorphism $\varphi : K \rightarrow \bar F$ extending the identity has the same image $\varphi(K)$
  \item[(iii)] Every irreducible $f(x) \in F[x]$ with one root in $\K$ has all its roots in $\K$
  \end{itemize}
\end{theorem}
\begin{proof}
  One implies two because if $\K$ is a splitting field for $P \subset F[x]$ then $\varphi(K)$ is the splitting field for $\varphi(P)$, which is just $P$ since the coefficients are in $P$. Hence this is independent of the homomorphism $\varphi$.
  Two implies three since given $\alpha \in K$ a root of $f(x)$ and any other root $\alpha' \in \bar K$ of $f(x)$, the iso extension theorem gives a map from $\bar K \rightarrow \bar K$ sending $\alpha$ to $\alpha'$, so since the image of this isomorphism contains $\alpha'$, and thus the original image under the identity also contains $\alpha'$.
  Three implies two since if three holds then $\K$ is the splitting field for the minimal $\F$ polynomial for all $\alpha \in K$, equivalently all of the $m_{F,\alpha}(x)$ with $\alpha \in K$.
\end{proof}
\begin{remark}
  This means it makes sense to think of splitting fields (normal extensions) as 'root closed' fields.
\end{remark}


\section*{Separability}

\begin{definition}
  A polynomial $f(x) \in \F[x]$ is separable if when we split it in some $\K$ it has distinct roots.
\end{definition}
\begin{example}
  $x^4 + x^2 + 1$ in $\F_2[x]$ is inseparable since it is just $(x^2 + x + 1)^2$. $x^2 + x + 1$ is irreducible and separable since in $\F_4$ it splits as $(x + \alpha)(x + \alpha + 1)$.
\end{example}
\begin{example}
  in $\F_p(t)[x]$, $f(x) = x^p - t$ is irreducible and inseparable, since in the extension $\F_p(t^{1/p})$ if we let $\alpha = t^{1/p}$ then $f(x) = x^p - t = x^p - \alpha^p = (x - \alpha)^p$, so $\alpha$ is a repeated root.
\end{example}
\begin{remark}
  $f$ is inseparable iff $gcd_{F[x]}(f,f') \ne 1$, since we've previously shown something even more general involving repeated roots. In particular an irreducible is inseparable iff it has identically zero derivative.
\end{remark}

\begin{definition}
  Say a field $\F$ is perfect if every irreducible in $\F[x]$ is seperable, equivalently if every irreducible has non-zero derivative.
\end{definition}
\begin{remark}
  characteristic $0$ fields are all perfect, finite fields are all perfect, and characteristic $p$ fields are perfect if the Frobenius endomorphism surjects.
\end{remark}

\begin{remark}[HW 13.6.9]
  $A$ is diagonalizable iff $m_{A,\F}(x)$ has no repeated roots (i.e.\ is separable), where $(m_{A,\F}(x)) = \ker(ev_A : \F[x] \rightarrow \F^{n \times n})$. 
\end{remark}


\section*{Cyclotomic extensions}
\begin{definition}
  $split_\Q(x^n - 1)$ is the $n$th cyclotomic extension of $\Q$. Note that the roots of $x^n - 1$ are the $n$th roots of unity in $\C$, which are denoted $1,\zeta_n, \zeta_n^2,...,\zeta_n^{n-1}$ and lie on the unit circle. 
\end{definition}
\begin{definition}
  The $n$th cyclotomic polynomial is defined to be $\Phi_n(x) = \prod_{\text{primitive nth roots of unity}}(x - \alpha) = \prod_{\alpha \in (\Z/n\Z)^\times} (x - \zeta_n^\alpha)$, which is the polynomial with the $n$th primitive roots of unity as all of its roots.
\end{definition}
\begin{theorem}
  $x^n - 1 = \prod_{d | n} \Phi_d(x)$ in $\C[x]$. Furthermore $\Phi_n(x)$ lies in $\Z[x]$, and is monic of degree $\varphi(n)$.
\end{theorem}
\begin{example}
  In the particular case of $p$ a prime, all of $\Z/n\Z[p]$ are units, so $\Phi_p(x) = \frac{x^p - 1}{x - 1}$.
\end{example}

\begin{theorem}
  $\Phi_n(x)$ is irreducible in $\Q[x]$, and so it is the minimal polynomail for $\zeta_n$. Hence the $n$th cyclotomic extension $\Q(\zeta_n)$ is a degree $\deg(\Phi_n(x)) = \varphi(n)$ extension of $\Q$.
\end{theorem}



\section*{Galois Theory}
The goal of Galois Theory is to understand finite extensions $\K/\F$ via their symmetries in $Aut(\K/\F)$, the automorphisms of $\K$ fixing $\F$.

\begin{remark}
  $Aut(K/F)$ sends roots $\alpha \in K$ of a polynomial $f \in F[x]$ to roots of $f$ since coefficients in $f$ are fixed by the automorphsim.
\end{remark}

\begin{example}
  The conjugation map is an element of $Aut(\C/\R)$ and $Aut(\Q/\C)$. It generates the first but not the second. $\R \Q \Z \C$.
\end{example}

\begin{example}
  Elements of $Aut(\Q(\zeta_n)/\Q)$ send $\zeta_n$ to other primitive roots of unity, so the automorphism group is isomorphic to $\Z/n\Z$.
\end{example}

\begin{definition}
  The fixed field $\K^H$ is the elements $k \in K$ such that $f(k) = k$ for all $f \in H$. This gives a correspondence between intermediate subfields $\F \subseteq \K^H \subseteq \K$ and subgroups $H = Aut(K/K^H) \le Aut(\K/\F)$.
\end{definition}
\begin{theorem}
  The following are true for finite extensions $\K/\F$:
  \begin{itemize}
  \item[(i)] $\F \subseteq \K^{Aut(\K/\F)}$
  \item[(ii)] $|Aut(\K/\F)| \le [\K : \F]$
  \end{itemize}
  And furthermore the following are equivalint definitions of a Galaois Extesnsion:
  \begin{itemize}
  \item[(a)] (i) is an equality
  \item[(b)] There exists some $G \le Aut(\K)$ for which $\F = \K^G$ (i.e. F is a fixed field)
  \item[(c)] (ii) is an equality
  \item[(d)] $\K = Split_\F(f(x))$ where $f(x)$ is a seperable polynomial in $\F[x]$
  \end{itemize}
\end{theorem}

\begin{theorem}
  If $\K/\F$ is Galois then there is a bijection between indtermediate subfields between $\K$ and $\F$ and intermediate subgroups between $Gal(\K/\F) = Aut(\K/\F)$. The bijections are given by $\K^-$ and $Aut(\K/-)$. Furthermore we have correspondences between normality of the groups and extensions being Galois. We also have correspondence between meets and joins in the two lattices, and have that the bijection is order reversing.
\end{theorem}

\begin{remark}
  The minimal polynomial for $\alpha \in \K$ is the polynomial whose roots are exactly the distinct images of $\alpha$ under automorphisms $g \in Gal(\K/\F)$. In particular this implies that elements of a Galois extension have seperable minimal polynomial.
\end{remark}


\end{document}
