%&pdflatex
\documentclass[11pt]{article}

\usepackage{main-macros}

\usepackage{newpxmath}
\usepackage{newpxtext}
\usepackage[margin=0.75in]{geometry}


\title{Ring Signatures on Hard Homogonous Spaces}
\author{Devon Tuma}
\date{\today}

\begin{document}
\maketitle


\section*{Background}

\subsection*{Motivation}

\cite{deniableAKE} gives a deniable authenticated key-exchange (AKE) with some quantum-transitional security, but is not a fully post-quantum secure protocol.
It relies on three main cryptographic primitives: a dual-reciever encryption function, a ring signature scheme, and a key encapsulation function.
The key encapsulation function is meant to provide quantum-transitional security, and so isn't needed when adapting the protocol to be post-quantum secure.
The encryption function can also be adapted through mostly direct translation to a post-quantum encryption scheme.
Finally, adapting the ring signature scheme motivates the creation of a ring signature scheme that applies to the keys of post-quantum key exchanges.

By further generalizing the ring-signature scheme to arbitrary group size, these ring signatures can also be applied to creating a post-quantum secure authenticated \textbf{group} key exchange (AGKE).
\cite{latticeAGKE} describes a non-deniable post-quantum AGKE based on the Ring-LWE problem.
The paper gives a non-authenticated GKE protocol, and applies the Katz-Yung compiler \cite{KYcompiler} to turn this into an AGKE.
This compiler takes a non-deniable signature based approach, and so the resulting AGKE is not deniable.
By modifying the compiler to use ring-signatures instead, the resulting Ring-LWE AGKE should be made deniable and still secure.

\subsection*{Hard Homogonous Spaces}

Hard homogonous spaces give a general context for a number of Diffie-Hellman style key exchanges, including some with post-quantum security.
Such a space is a commutative group $G$ and a transitive $G$-action on some set $X$, with constraints on computational complexities of certain operations.
For example $G$-multiplication and the $G$-action should both be efficiently computable, but the discrete logarithm problem (given $x,x' \in X$, find $g \in G$ with $x' = g * x$) should be infeasible.
Such a space gives a natural Diffie-Hellman key exchange: Alice and Bob choose secret keys $a$ and $b$ from $G$, exchange $a * x$ and $b * x$ for some fixed $x \in X$, and then compute a shared secret $a * (b * x) = (ab) * x = b * (a * x)$.
\cite{CSIDH} describes the post-quantum secure CSIDH protocol and has provides a full definition and discussion of HHS.

\section*{Ring Signature Scheme}

The ring signature scheme used in \cite{deniableAKE} is based on traditional Diffie-Hellman on a cyclic group $C$ of order $p$.
This corresponds to the HHS where $X$ is the set of generators of $C$, and $G$ is the multiplicative group $(\Z/p\Z)^\times$ acting via exponentiation.
However, direct translation to the HHS setting isn't possible, because the ring signature scheme uses both the multiplicative structure and the additive structure of $(\Z/p\Z)^\times$, which doesn't exist in the generalized setting.
This essentially creates a dependence on classical Diffie-Helman as a 'ring action' rather than a 'group action'.

However, we can modify the scheme to return to just the 'group action' situation, assuming there is an efficient hash function into the group $G$ (it isn't clear to me if this is always possible).
In this case we get the following adapted ring signature protocol ($x_0$ is a fixed generator of $X$, $*$ denotes the group action of $G$ on $X$, $\times$ denotes multiplication in $G$, and $A_i = a_i * x_0$ is the public key assosiated to the secret key $a_i$, and the resulting proof is that the signer knows one of the secret keys in $\{a_1, a_2, a_3\}$):

\begin{itemize}
\item [1] Generate random values $t_1, c_2, c_3, r_2, r_3 \in G$
\item [2] Compute $T_1 = t_1 * x_0$.
\item [3] Compute $T_2 = (r_2 \times c_2) * A_2$ and $T_3 = (r_3 \times c_3) * A_3$.
\item [4] Compute $c = H(x_0 | A_1 | A_2 | A_3 | T_1 | T_2 | T_3 | m)$, where $H$ is a hash function into the group $G$ and $m$ is the message being signed.
\item [5] Compute $c_1 = c \times c_2^{-1} \times c_3^{-1}$.
\item [6] Compute $r_1 = t_1 \times a_1^{-1} c_1^{-1}$.
\end{itemize}

The resulting proof consists of $(c_1, r_1, c_2, r_2, c_3, r_3)$. To verify the proof, the verifier computes $c' = H(x_0 | A_1 | A_2 | A_3 | (r_1 \times c_1) * A_1 | (r_2 \times c_2) * A_2 | (r_3 \times c_3) * A_3 | m)$ and checks whether $c' = c_1 + c_2 + c_3$. This is equivalent to checking that $T_1 = (r_1 \times c_1) * A_1$, although the verifier can't check this directly since $T_1$ is never directly published.

The security proofs can also be modified to fit the more generalized setting. It should also be possible to formalize the security proofs in something like \cite{formalize}, which I've started looking at doing.
Expanding the ring signature to secret keys $\{a_1, \dots, a_n\}$ can be done directly, but the signature size grows linearly in the number of secret keys. I think it might be possible to reduce this but I haven't looked very closely yet.

\bibliography{deniable-AGKE-compiler}{}
\bibliographystyle{plain}

\end{document}

