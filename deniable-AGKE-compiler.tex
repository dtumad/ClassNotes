%&pdflatex
\documentclass[11pt]{article}

\usepackage{main-macros}

\usepackage{newpxmath}
\usepackage{newpxtext}
\usepackage[margin=0.75in]{geometry}


\title{Deniable post-quantum authenticated key exchanges}
\author{Devon Tuma}
\date{Fall 2020}

\begin{document}
\maketitle


\section*{Motivation}

\cite{deniableAKE} gives a deniable authenticated key-exchange (AKE) with some quantum-transitional security, but is not a fully post-quantum secure protocol.
It relies on three main cryptographic primitives: a dual-reciever encryption function, a ring signature scheme, and a key encapsulation function.
The key encapsulation function is meant to provide quantum-transitional security, and so isn't needed when adapting the protocol to be post-quantum secure.
The encryption function can also be adapted through mostly direct translation to a post-quantum encryption scheme.
Finally, adapting the ring signature scheme motivates the creation of a ring signature scheme that applies to the keys of post-quantum key exchanges.

By further generalizing the ring-signature scheme to arbitrary group size, these ring signatures can also be applied to creating a post-quantum secure authenticated \textbf{group} key exchange (AGKE).
\cite{latticeAGKE} describes a non-deniable post-quantum AGKE based on the Ring-LWE problem.
The paper gives a non-authenticated GKE protocol, and applies the Katz-Yung compiler \cite{KYcompiler} to turn this into an AGKE.
This compiler takes a non-deniable signature based approach, and so the resulting AGKE is not deniable.
By modifying the compiler to use ring-signatures instead, the resulting Ring-LWE AGKE should be made deniable and still secure.

\section*{HHS Ring Signature in ROM}

\subsection*{Hard Homogonous Spaces}

Hard homogonous spaces(HHS) give a general context for a number of Diffie-Hellman style key exchanges, including those with post-quantum security.
Such a space is just a group commutative $G$ and a transitive $G$-action on some set $X$, with constraints on computational complexits of certain operations.
For example $G$-multiplication and the $G$-action should both be efficiently computable, but the discrete logarithm problem (given $x,x' \in X$, find $g \in G$ with $x' = g * x$) should be infeasible.
Such a space gives a natural Diffie-Hellman key exchange: Alice and Bob choose secret keys $a$ and $b$ from $G$, exchange $a * x$ and $b * x$ for some fixed $x \in X$, and then compute a shared secret $a * (b * x) = (ab) * x = b * (a * x)$.
\cite{CSIDH} describes the post-quantum secure CSIDH protocol and has provides a full definition and discussion of HHS.

\subsection*{Ring Signature Scheme}

The ring signature scheme used in the deniable protocol of \cite{deniableAKE} is based on traditional Diffie-Hellman on a cyclic group $C$ of order $p$.
This corresponds to the HHS where $X$ is the set of generators of $C$, and $G$ is the multiplicative group $(\Z/p\Z)^\times$ acting via exponentiation.
However, direct translation to the HHS setting isn't possible, because the ring signature scheme uses both the multiplicative structure and the additive structure of $(\Z/p\Z)^\times$, which doesn't exist in the generalized setting.
This essentially creates a dependence on classical Diffie-Helman as a 'ring action' rather than a 'group action'.

However, we can modify the scheme to return to just the 'group action' situation, assuming the inverse operation in the group is efficiently computable, via the following modified protocol:

\begin{itemize}
\item[1] Generate random values $t_1, c_2, r_2, c_3, r_3 \in G$
\item [2] Do the rest
\end{itemize}

The security proofs can also be modified appropriately to fit the more generalized setting.


\bibliography{deniable-AGKE}{}
\bibliographystyle{plain}

\end{document}

