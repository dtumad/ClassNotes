%&pdflatex
\documentclass[11pt]{article}

\usepackage{main-macros}

\usepackage{newpxmath}
\usepackage{newpxtext}
\usepackage[margin=0.75in]{geometry}


\title{8271 10/07 Reading}
\author{Devon Tuma}
\date{Fall 2020}

\begin{document}
\maketitle

\section*{Question Answering}

\begin{itemize}
\item [1] What are the three major risks about DockerHub presented in the paper?

  Firstly, sensitive parameters in commands can expose information such host-files and display.
  Secondly, malicious images can have remote execution or malicious crypto-mining attacks built in.
  Finally, vulnerability patching of Docker images is slow and delayed.
  
\item [2] Why is the patching of the vulnerabilities in Docker images often delayed or even ignored?

  Programs inside of Docker images are decoupled from the actual mainstream release version of these programs.
  This is necessary in order to make images stable and self-contained, but there can be little incentive for developers of images to update the internal programs.
  More generally, there also tends to be much less frequent updates to Docker images than general programs, so even if each update patches the programs this can leave a significant security gap between image updates.
  Security patches to the mainstream program can therefore be missing from the versions of the programs being run in Docker images.
  Unmaintained Docker images that are still be used can also contribute to this phenomenon.
\end{itemize}

\section*{Paper Critiques}

\subsection*{Short Summary}

The paper focuses on analyzing security risks of Docker image repositories, in particular Docker Hub.
It focuses on three vectors of attack for this system, and analyzes prevalence of existing issues, as described above.
It also looks at some potential mitigations for the analyzed vulnerabilities, although it doesn't examine actual implementation of these mitigations.

\subsection*{Limitations of the paper}

\begin{itemize}
\item The paper focuses on existing attacks and issues, without significant analysis of potential attack vectors that may exist even if they aren't currently massively exploited
\item The paper focuses on Docker Hub very specifically, but some of the findings could be applied to general online sources of software
\end{itemize}

\subsection*{Solutions to the limitations}

The second limitation could be addressed by looking at other types of online repositories and looking at how these vulnerabilities apply to those situations.

\subsection*{Potential follow-up work}

One follow-up work would be looking at actually implementing mitigations for the vulnerabilities that are being considered.
In particular, this would allow analysis of actual level of effectiveness of the various approaches.

\end{document}
