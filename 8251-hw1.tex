%&pdflatex
\documentclass[11pt]{article}

\usepackage{main-macros}

\usepackage{newpxmath}
\usepackage{newpxtext}
\usepackage[margin=0.75in]{geometry}

\newcommand{\disc}{\text{disc}} 

\title{8251 HW1}
\author{Devon Tuma}
\date{Fall 2020}

\begin{document}
\maketitle

\problem{1.16.}
Show that $(1 - \omega)(1 - \omega^2)\dots(1 - \omega^{p-1}) = p$.

\begin{proof}
  Setting $g(t) := (t - \omega)\dots(t - \omega^{p-1})$, we want to show that $g(1) = p$.
  From equation (1.2) we have that
  \begin{equation*}
    t^p - 1 = (t - 1)(t - \omega)\dots(t - \omega^{p-1}) = (t-1)g(x)
  \end{equation*}
  If we take formal derivatives of both sides, we get that $pt^{p-1} = g(x) + (t-1)g'(x)$, and evaluating this at $1$ we see that $p = g(1) + 0 * g'(1) = g(1)$, which is exactly what we wanted to show.
\end{proof}


\problem{1.21}
Show that every member of $\Q[\omega]$ is uniquely representable in the form $$a_0 + a_1\omega + \dots + a_{p-2}\omega^{p-2}, \ a_i \in \Q$$

\begin{proof}
  We need to show that $\{1, \omega, \dots, \omega^{p-2}\}$ is a $\Q$-basis for $\Q[\omega]$.
  
  To show that it is a spanning set, it suffices to show $\omega$ is a root of $f(t) = t^{p-1} + \dots + t + 1$, since then $\omega^{p-1} = -\omega^{p-2}-\dots-t-1$ and any power $\omega^m$ with $m \ge p - 1$ can be recursively reduced to a smaller power by substituting this equation.
  But note that we have $\omega f(\omega) = 1 + \omega^{p-1} + \dots + \omega^2 + \omega = f(\omega)$, and $\omega \ne 1$, so we must have $f(\omega) = 0$ as desired.

  Then to show the set is linearly independent, it suffices to show $f(t)$ is irreducible, since any nontrivial linear dependence among the $\{1, \omega, \dots, \omega^{p-2}\}$ defines a polynoial with $\omega$ as a root, of degree stricly less than $p-1$, which is then a nontrivial factor of $f(t)$.
  Furthermore, it suffices to show $f(t+1)$ is irreducible, since $g(t) |\ f(t)$ iff $f(t+1) |\ g(t+1)$. Then note that $(t-1)f(t) = t^p - 1$, so that $f(t+1) = ((t+1)^p - 1) / t)$.
  Applying the binomial thoerem, we see $f(t+1) = \sum_{n = 1}^p \binom{p}{n} t^{n-1}$.
  But then all coefficients but the first coefficient are divisible by $p$, and the last coefficient is exactly $p$ which isn't divisible by $p^2$, so we can apply Eisenstein's criterion to get irreducibility as desired.
\end{proof}

  
\problem{1.22}
Show that if $\alpha \in \Z[\omega]$ and $p |\ \alpha = \alpha_0 + \alpha_1\omega + \dots + \alpha_{p-2}\omega^{p-2}$, then all the $\alpha_i$ are divisible by $p$.

\begin{proof}
  If we have divisiblity, let $\beta$ be such that $p\beta = \alpha$.
  Since irreducibility in $\Q[x]$ implies irreducibility in $\Z[x]$, the previous exercise also implies a similar representation for elements of $\Z[x]$.
  In particular, $\beta$ has a representation of the form $\beta = \beta_0 + \beta_1\omega + \dots + \beta_{p-2}\omega^{p-2}$ with $\beta_i \in \Z$.
  But then by the linear independence of the powers of $\omega$, we must have $\alpha_i = p\beta_i$, and we conclude that $p | \alpha_i$ for all $i$.
\end{proof}

\problem{2.4}
Suppose $a_0, \dots, a_{n-1}$ are algebraic integers and $\alpha$ is a complex number satisfying $\alpha^n + a_{n-1}\alpha^{n-1} + \dots + a_0 = 0$. Show taht the ring $\Z[a_0, \dots, a_{n-1}, \alpha]$ has finitely generated additive group.

\begin{proof}
  We know that elements of $\Z[a_0, \dots, a_{n-1}, \alpha]$ are finite $\Z$-linear combinations of products of the form $a_0^{k_0} \dots a_{n-1}^{k_{n-1}} \alpha^k$, so it suffices to show that only many of these products are independent.
  By assumption, we have $\alpha^n = -a_{n-1}\alpha^{n-1} - \dots - a_0$, so we can certainly reduce any term with a power of $\alpha$ greater than $n$ to a linear combination of terms with powers of $\alpha$ strictly less than $n$ by recursively substituting this equation (each substitution strictly reduces the maximal power of $\alpha$ so this process terminates).

  But we also have that each $a_i$ is an algebraic integra, so there exists monic polynomials $p_i \in \Z[x]$ such that $p_i(a_i) = 0$ for all $i$.
  Each $p_i$ is monic, so is of the form $x^{deg(p_i)} + p'_i(x)$ where the trailing terms in $p'_i$ have degree strictly less than $deg(p_i)$.
  Hence we have $a_i^{deg(p_i)} = - p'_i(a_i)$, where each $p'_i(a_i)$ is a $\Z$-linear combination of powers of $a_i$, with each power of $a_i$ having degree less than $deg(p_i)$.
  So again we can recursively substitute these values, and the process will again eventually terminate since each substituion strictly reduces the maximal degree of a power of $a_i$.

  Hence after all of the substitutions, we can write any element as a linear combination of terms of the form $a_0^{k_0} \dots a_{n-1}^{k_{n-1}} \alpha^k$, with $k_i < deg(p_i)$ and $k < n$, and so this forms a finite generating set as desired.
  Then since $\alpha \Z[a_0, \dots, a_{n-1}, \alpha] \subset \Z[a_0, \dots, a_{n-1}, \alpha]$, we finally conclude that $\alpha$ is an algebraic integer as well.
\end{proof}


\problem{2.6}
Show that if $f$ and $g$ are polynomials over a field $K$ and $f^2 |\ g$ in $K[x]$ then $f |\ g'$.

\begin{proof}
  Since $f^2 |\ g$, we have $g = f^2h$ for some $h \in K[x]$.
  Taking formal derivatives, we have $$g' = 2ff'h+f^2h' = f(2f'h+fh')$$ and we see that $f |\ g'$ exactly as desired.
\end{proof}


\problem{2.10}
Show that if $m |\ r$ and $\varphi(r) \le \varphi(m)$ then $r = m$.

\begin{proof}
  Since $m |\ r$, we have $m \le r$, and so it suffices to show $r \le m$.

  Let $m = p_0^{e_0} p_1^{e_1} \dots p_n^{e_n}$ be the prime factorization of $m$.
  Then since $m |\ r$, the prime factorization of $r$ must be of the form $r = k * p_0^{e_0 + d_0} p_1^{e_1 + d_1} \dots p_n^{e_n + d_n}$ for some $k$ relatively prime to all of the $p_i$.
  Using the standard identities for $\varphi$, we then get:
  \begin{equation*}
    \varphi(m) = \prod_{0 \le i \le a} (p_i^{e_i} - p_i^{e_i - 1}) \le
    \varphi(k) \prod_{0 \le i \le a} (p_i^{e_i + d_i} - p_i^{e_i + d_i - 1})
    = \varphi(r)
  \end{equation*}
  And we already had $\varphi(m) \ge \varphi(r)$, so we have in fact $\varphi(m) = \varphi(r)$.
  But then looking at the above equation egain, we see that equality holds only if $d_i = 0$ for all $i$ and $\varphi(k) = 1$.
  $\varphi(k) = 1$ implies $k = 1$, and then if we plug these values back into the oringal prime factorizations we see that they are now identical, and we conclude $r = m$ as desired.
\end{proof}


\problem{2.25}
Show that for any algebraic number $\alpha$, there exists $m \in \Z$, $m \ne 0$, such that $m\alpha$ is an algebraic integer.

\begin{proof}
  Let $\alpha$ be algebraic, and $p \in \Q[x]$ a witness to this, so $p(\alpha) = 0$.
  Clearing denominators on the coefficients of $p$, we can write $p = kf$, for some $f \in \Z[x]$ and $k \in \Q$.
  Note that $f(\alpha) = p(\alpha)/k = 0$, so $\alpha$ is a root of $f$.
  Hence $0 = f(\alpha) = c_n\alpha^n + \dots + c_1\alpha + c_0$, where the coefficeients $c_i$ are all integers.
  We will show $c_n\alpha$ is integral ($c_n$ is non-zero since $p$, and hence $f$, is nontrivial).
  
  Multiplying the equation by $c_n^{n-1}$ we get:
  \begin{align*}
    0 = c_n^{n-1} f(\alpha)
    &= c_n^n\alpha^n + c_{n-1}c_n^{n-1}\alpha^{n-1} + c_{n-2}c_n^{n-1}\alpha^{n-2} + \dots + c_0c_n^{n-1} \\
    &= (c_n\alpha)^n + c_{n-1}(c_n\alpha)^{n-1} + c_{n-2}c_n(c_n\alpha)^{n-2} + \dots + c_0c_n^{n-1}
  \end{align*}
  And we see that $c_n\alpha$ is a root of the polynomial $x^n + c_{n-1}x^{n-1} + c_{n-1}c_nx^{n-2} + \dots + c_0 c_n^{n-1}$, which is monic and has integer coefficients.
  Hence $c_n\alpha$ is an algebraic integer, as desired.
\end{proof}


\problem{2.28}
Let $f(x) = x^3 + ax + b$, with $a,b \in \Z$ and assume $f$ is irreducible over $\Q$ with $\alpha$ a root of $f$.

\textbf{(a)} Show that $f'(\alpha) = -(2a\alpha + 3b)/\alpha$.

\begin{proof}
  We have $f'(x) = 3x^2 + a$, so $f'(\alpha) = 3\alpha^2 + a = (3\alpha^3 + a\alpha)/\alpha$.
  But since $\alpha$ is a root of $f$, we also know $\alpha^3 = -a\alpha - b$, and substituting this into the previous equation gives $f'(\alpha) = -(3a\alpha - 3b - a\alpha)/\alpha = -(2a\alpha + 3b)/\alpha$ as desired.
\end{proof}

\textbf{(b)} Show that $2a\alpha+3b$ is a root of $(\frac{x-3b}{2a})^3+a(\frac{x-3b}{2a})+b$, Use this to find $N_\Q^{\Q[\alpha]}(2a\alpha+3b)$.

\begin{proof}
  Note that the given equation is just $f(g(x))$ where $g(x):=(x-3b)/2a$.
  Furthermore, we have $g(2a\alpha+3b)=2a\alpha/2a=\alpha$, so we have $f(g(2a\alpha+3b)) = f(\alpha) = 0$ by definition of $\alpha$.
  Hence $2a\alpha+3b$ is a root of $f \circ g$ as desired.
\end{proof}

  Then this must be the minimal polynomial for $2a\alpha+3b$ since we assumed $f$ was irreducible, so to find $N_\Q^{\Q[\alpha]}(2a\alpha+3b)$ we just need to find the product of all its roots (since these roots are exactly the conjugates).
  But note that after normalizing by the leading coefficient to make the polynomial monic, this is just the constant term of the resulting polynomial.
  The leading coefficient is $1/8a^3$, and the constant term is $-27b^3/8a^3-3b/2+b$, so we conclude that $N_\Q^{\Q[\alpha]}(2a\alpha+3b) = -27b^3-12ba^3 + 8ba^3 = -27b^3-4ba^3$.

\textbf{(c)} Show that $\disc(\alpha) = -(4a^3+27b^2)$

\begin{proof}
  By theorem 8, we have $\disc(\alpha) = N(f'(\alpha))$.
  Then by part (a) we know $f'(\alpha) = -(2a\alpha + 3b)/\alpha$, so using multiplicativity of the norm, we have $\disc(\alpha) = -N(2a\alpha + 3b)/N(\alpha)$.
  Furthermore, we know $N(\alpha) = b$ since $f$ is irreducible and monic, and plugging in this and the result of (b) gives $\disc(\alpha) = (-27b^3-4ba^3)/b = -(4a^3 + 27b^2)$ as desired.
\end{proof}

\textbf{(d)} Suppose $\alpha^3 = \alpha + 1$. Prove that $\{1,\alpha,\alpha^2}$ is an integral basis for $\mathbb{A} \cap \Q[\alpha]$.

\begin{proof}
  If $\alpha^3 = \alpha + 1$ then $\alpha$ is a root of $x^3 - x - 1$. This is irrudicible since it has no roots, so we can apply (c) to get $\disc(\alpha) = -23$.
  Since this is squarefree, we can apply 27.e to get that $\{1,\alpha,\alpha^2\}$ is an integral basis.
  If instead $\alpha^3 + \alpha = 1$, then $\alpha$ is instead a root of $x^3 + x - 1$, so $\disc(\alpha) = -31$, which is again squarefree so we have the same integral basis.
\end{proof}


\problem{2.37}
Let $\alpha$ be an algebraic number of degree $n$ over $\Q$ and lef $f$ and $g$ be two polynomials over $\Q$, each of degree < $n$, such that $f(\alpha) = g(\alpha)$. Show that $f = g$.

\begin{proof}
  Since $f(\alpha) = g(\alpha)$, we have $(f - g)(\alpha) = 0$.
  Furthermore, we have $deg(f-g) \le max\{deg(f),\ deg(g)\} < n$.
  But $\alpha$ is of degree $n$, so it can't be a solution to a non-trivial polynomial of degree strictly less than $n$.
  Therefore we conclude $f - g = 0$ and so finally $f = g$ as desired.
\end{proof}




\end{document}
